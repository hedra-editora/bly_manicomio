%!TEX root=./LIVRO.tex


\part{Dez Dias em um Manicômio}

%\imagemmedia{}{./media/rId21.jpg}

% Nellie Bly

% Tradução: Francisco Araujo da Costa

\chapter{Introdução}\label{introduuxe7uxe3o}

Desde que minhas experiências no Hospício da Ilha de Blackwell foram
publicadas no \emph{New York World}, recebi centenas de cartas sobre o
caso. A edição com a minha história se esgotou há bastante tempo, então
fui convencida a permitir que ela fosse publicada em livro para
satisfazer as centenas que ainda procuram um exemplar.

Fico feliz em informar que, devido à minha visita ao hospício e as
reportagens subsequentes, a Cidade de Nova Iorque reservou uma soma
1.000.000 de dólares por ano maior do que jamais em sua história para o
cuidado dos insanos. Assim, pelo menos tenho a satisfação de saber que
as pobres infelizes receberão cuidados melhores por causa do meu
trabalho.

\openany  



\imagemmedia{}{./media/rId24.png}
\chapter{Uma missão
delicada}\label{capuxedtulo-i-uma-missuxe3o-delicada}

No dia 22 de setembro, o \emph{New York World} perguntou se eu poderia ser
internada em um dos hospícios de Nova Iorque, com a ideia de escrever
uma narrativa simples e sem enfeites do tratamento que os pacientes
recebem nele, os métodos de administração e assim por diante. Eu teria
coragem para enfrentar a provação exigida por essa missão? Conseguiria
assumir as condições de insanidade a ponto de enganar os médicos e viver
entre os loucos por uma semana sem que as autoridades do local
descobrissem que eu era apenas uma observadora disfarçada? Respondi que
acreditava que sim. Eu tinha alguma fé na minha capacidade como atriz e
achava que poderia fingir insanidade pelo tempo necessário para cumprir
qualquer missão que me fosse dada. Eu poderia passar uma semana na ala
dos insanos na Ilha de Blackwell? Disse que podia e passaria. E passei.

Minhas instruções foram simplesmente começar o meu trabalho assim que me
sentisse pronta. Eu deveria escrever uma crônica fiel de minhas
experiências e, depois de entrar no hospício, descobrir e descrever seu
funcionamento interno, sempre tão bem escondido do conhecimento do
público pelas enfermeiras de boinas brancas, além das grades e
fechaduras. @--- Não estamos pedindo que vá até lá para fazer revelações
sensacionais. Escreva o que descobrir, bom ou ruim. Distribua os elogios
ou a culpa que achar melhor e conte sempre a verdade. Mas tenho receio
desse seu sorriso crônico --- meu editor disse. @--- Então não vou mais
sorrir --- respondi, e saí para executar minha delicada e, como
descobriria, difícil missão.

Se conseguisse entrar no hospício, o que não tinha muita esperança de
conseguir, eu não imaginava que minhas experiências passariam de uma
simples história sobre a vida em um hospício. Que tal instituição
pudesse ser mal administrada, que poderia haver crueldades sob seu teto,
eu não considerava possível. Sempre tive o desejo de conhecer a vida dos
hospícios mais completamente, o desejo de ser convencida de que as
criaturas mais indefesas de Deus, os loucos, recebiam cuidados bondosos
e corretos. As muitas histórias de abusos nessas instituições eu
considerava exageros absurdos ou então mentiras, mas ainda tinha o
desejo latente de ter certeza.

Eu tremia só de pensar em como os loucos estavam completamente à mercê
de seus guardiões e como era possível chorar e implorar por liberdade,
sem nenhum resultado, se os guardiões assim quisessem. Aceitei
ansiosamente a missão de descobrir como era o funcionamento interno do
Hospício da Ilha de Blackwell.

--- Como você vão me tirar de lá --- perguntei a meu editor --- depois
que eu entrar?

--- Não sei --- ele respondeu. --- Mas vamos tirá-la de lá nem que
tenhamos que contar quem você é e por que fingiu insanidade. Só entre.

Eu tinha pouca fé na minha capacidade de enganar os especialistas em
insanidade. Creio que meu editor tinha ainda menos.

Todas as preparações preliminares para o meu martírio ficaram a meu
encargo. Apenas uma decisão foi tomada antes, a saber, que eu fingiria
loucura sob o pseudônimo de Nellie Brown, usando as mesmas iniciais do
meu nome e bordadas na minha roupa de baixo, para que não houvesse
dificuldade em acompanhar meus movimentos e me salvar das dificuldades
ou perigos que poderia enfrentar. Havia meios de entrar na ala dos
insanos, mas eu não os conhecia, então tinha duas opções a meu dispor.
Eu poderia fingir insanidade na casa de amigos e ser internada com base
na decisão de dois médicos competentes ou então poderia atingir meu
objetivo usando o tribunal de polícia.

\imagemmedia{}{./media/rId25.png}Depois
de alguma reflexão, decidi que seria mais sábio não criar essa imposição
para meus amigos ou encontrar um médico de boa índole disposto a me
auxiliar nesse plano. Além disso, para ser enviada à Ilha de Blackwell,
meus amigos precisariam fingir pobreza e, infelizmente para o objetivo
que tinha em vista, meu contato com os mais pobres, com exceção de eu
mesma, era apenas superficial. Assim, me decidi pelo plano que levou ao
sucesso de minha missão. Consegui ser internada na ala dos insanos da
Ilha de Blackwell, onde passei dez dias e dez noites e tive uma
experiência que jamais esquecerei. Assumi a missão de interpretar uma
pobre louca infeliz e acreditava que seria meu dever não me esquivar de
nenhum dos resultados desagradáveis desse processo. Fiquei sob a tutela
da cidade durante todo o período, tive diversas experiências e vi e ouvi
muito sobre o tratamento dado a essa classe indefesa de nossa população.
Depois que vi e ouvi o suficiente, minha liberdade foi obtida
imediatamente. Deixei a ala dos insanos com prazer e arrependimento:
prazer porque mais uma vez poderia respirar o ar livre dos céus,
arrependimento por não conseguir levar comigo algumas das infelizes que
viveram e sofreram ao meu lado e que acredito firmemente serem tão sanas
quanto eu era e sou.

Mas antes gostaria de afirmar uma coisa: Desde o instante em que entrei
na ala dos insanos na ilha, não fiz nenhuma tentativa de manter o papel
de insanidade que assumira. Falei e agi como faço na minha vida normal.
Por mais estranho que pareça, quanto mais sãs eram minhas ações e minha
conversa, mais louca eu era considerada por todos, exceto por um médico,
cuja bondade e gentileza jamais esquecerei.


\chapter{Preparação para a
provação}\label{capuxedtulo-ii-preparauxe7uxe3o-para-a-provauxe7uxe3o}

Mas voltemos a meu trabalho e minha missão. Após receber minhas
instruções, voltei a minha pensão e, com o cair da noite, comecei a
ensaiar o papel no qual faria minha estreia no dia seguinte. Que tarefa
mais difícil, pensei, se apresentar perante uma multidão e convencê-la
de que eu era louca. Nunca havia me aproximado de pessoas insanas antes
na vida e não fazia ideia de como seriam suas ações. E depois eu seria
examinada por vários médicos instruídos, homens que se especializam em
insanidade e que estão em contato diário com os loucos! Que chance eu
teria de enganar esses médicos e convencê-los de que era maluca? Meu
medo é que eles não poderiam ser tapeados. Comecei a pensar que minha
tarefa estava fadada ao fracasso. Ainda assim, ela precisava ser
tentada, então corri para o espelho e examinei meu rosto. Lembrei de
tudo que havia lido sobre as ações dos loucos, como antes de mais nada
todos têm um olhar fixo, então arregalei meus olhos ao máximo e encarei
meu reflexo sem piscar. Posso garantir-lhes que a visão não era nada
tranquilizadora, mesmo para mim mesma, especialmente no meio da
madrugada. Tentei aumentar o gás, na esperança de ganhar coragem. Tive
algum sucesso, mas me consolei com a ideia de que em poucas noites não
estaria mais lá, e sim presa em uma cela com um bando de lunáticas.

O tempo não estava frio, mas quando pensei no que estava por vir, senti
calafrios correndo pelas costas, em uma paródia da perspiração que se
pouco a pouco acabava com os cachos da minha franja. Entre ensaiar na
frente do espelho e imaginar meu futuro como lunática, eu lia passagens
de histórias de fantasmas improváveis e impossíveis, de modo que quando
o sol veio para expulsar a noite, senti que tinha o humor correto para
minha missão, mas ainda faminta o suficiente para saber que queria meu
café da manhã. Lenta e tristemente, tomei meu banho matinal e me despedi
de alguns dos artigos mais preciosos da civilização moderna. Guardei
minha escova de dentes com carinho e, quando me esfreguei pela última
vez com o sabonete, murmurei: @--- Pode ser por alguns dias e pode ser
por... por mais tempo. @ A seguir, vesti as roupas velhas que havia
selecionado para a ocasião. Estava com o espírito de analisar tudo pela
ótica da mais absoluta seriedade. Eu fazia muito bem em me despedir,
decidi, pois quem saberia dizer o que a tensão de me fingir de louca e
ficar presa com um grupo de malucas faria com meu próprio cérebro. Eu
poderia nunca mais voltar. Mas jamais cogitei fugir de minha missão. Com
muita calma, ou pelo menos parecendo calma, comecei minha loucura.

Primeiro decidi que seria melhor ir até uma pensão e, após obter um
quarto, confessar à senhoria (ou senhorio, dependendo de quem fosse) que
estava em busca de emprego e, alguns dias depois, aparentemente
enlouquecer. Quando reconsiderei a ideia, fiquei com medo que ela
demoraria demais para dar frutos. Foi então que decidi que seria muito
mais fácil ir até uma pensão para trabalhadoras. Eu sabia que se fizesse
uma casa cheia de mulheres acreditar que era maluca, elas não
descansariam até eu estar longe delas e presa em um lugar seguro.

A partir de um diretório, selecionei o Lar Temporário para Mulheres, no
número 84 da Segunda Avenida. Caminhando pela avenida, resolvi que
depois de chegar ao Lar, faria todo o possível para dar início à minha
jornada até a Ilha de Blackwell e ao Hospício.

\chapter{No lar
temporário}\label{capuxedtulo-iii-no-lar-temporuxe1rio}

Era chegado o momento de começar minha carreira como Nellie Brown,
insana. Caminhando pela avenida, tentei assumir o olhar das donzelas em
imagens intituladas ``Sonhando''. Expressões distantes têm ares de
loucura. Atravessei o pátio de concreto até a entrada do Lar e toquei a
campainha, que era tão alta quanto os sinos de uma igreja. Fiquei
aguardando nervosa a abertura da porta. Minha intenção era, em pouco
tempo, ser expulsa através dela, levada pela caridade da polícia. A
porta se escancarou com força e uma menina loira baixinha, de não mais
do que treze verões, apareceu.

--- A governanta está? --- perguntei baixinho.

--- Sim, ela está, mas está ocupada. Pode ir para a sala de estar dos
fundos --- a menina respondeu alto, sem demonstrar nenhuma alteração no
rosto estranhamente maduro.

\imagemmedia{}{./media/rId29.png}Segui
essas instruções, não muito bondosas ou educadas, e fui parar em uma
sala escura e desconfortável, onde fiquei esperando por minha anfitriã.
Depois de passar pelo menos vinte minutos sentada, uma mulher magra,
usando um vestido preto simples, apareceu na minha frente. @--- E então?
--- ela exclamou.

--- A senhora é a governanta? --- perguntei.

--- Não. A governanta está doente, eu sou a assistente. O que você quer?

--- Quero ficar aqui por alguns dias, se vocês tiverem vaga para mim.

--- Bem, não temos quartos individuais, estamos lotadas, mas se estiver
disposta a ocupar um quarto com outra garota, isso eu posso fazer por
você.

--- Fico muito agradecida --- respondi. --- Quanto vocês cobram? @ Eu
trouxera apenas setenta centavos comigo, sabendo muito bem que quanto
mais cedo meus fundos se exaurissem, mais cedo seria expulsa, e ser
expulsa era o que queria.

--- Cobramos trinta centavos por noite --- foi sua resposta à minha
pergunta. @ Com isso, paguei por uma estadia e ela me abandonou, dizendo
que tinha outro assunto para resolver. Deixada para me entreter como
pudesse, fiz uma avaliação do meu ambiente.

Ele não era nada alegre. Guarda-roupa, escrivaninha, estante de livros,
órgão e diversas cadeiras completavam a decoração da sala, na qual a luz
do sol mal entrava.

Quando terminei de me familiarizar com meus aposentos, um sino,
concorrente da campainha em termos de altura, começou a retinir no
porão, e ao mesmo tempo as mulheres começaram a marchar escada abaixo de
todas as partes da casa. Imaginei, pelos sinais óbvios, que o almoço
estava servido, mas como ninguém me disse nada, não me esforcei para
seguir o batalhão das famintas. Contudo, ainda queria que alguém me
convidasse para descer. Sempre sinto saudade de casa quando sei que os
outros estão comendo e não tive a chance, mesmo se não estou com fome.
Fiquei feliz quando a assistente da governanta subiu e me perguntou se
não queria comer. Respondi que queria e então perguntei qual era seu
nome. sra. Stanard, ela respondeu, e imediatamente anotei o nome no
caderninho que levara para fazer anotações e o qual preenchera com
várias páginas de baboseira para os cientistas que o encontrassem.

Assim preparada, passei a aguardar novos acontecimentos. Meu almoço, no
entanto... me dirigi à escadaria, que não tinha tapetes ou carpetes, e
segui a sra. Stanard até o porão, onde um grande número de mulheres
estava comendo. Ela abriu um lugar na mesa para mim junto com três
mulheres. A menininha de cabelos curtos que abrira a porta apareceu,
trabalhando de garçonete. Com as mãos na cintura, me olhando
envergonhada, perguntou:

--- Ensopado de carneiro, ensopado de carne de gado, feijão, batata,
café ou chá?

--- Carne de gado, batata, café e pão --- respondi.

--- Pão entra --- ela explicou, dirigindo-se para a cozinha, que ficava
nos fundos. @ Não demorou para que ela voltasse com o que eu havia
pedido em uma bandeja grande e amassada, que atirou na minha frente.
Comecei minha refeição simples. Não era muito apetitosa, então fingi
comer e comecei a observar as outras.

Muito dei sermões sobre a forma repulsiva que a caridade sempre assume.
Aqui estava eu, em um lar para mulheres dignas, e que piada era o nome.
O piso estava nu e as mesinhas de madeira ignoravam totalmente qualquer
acessório moderno, como verniz, polimento ou toalha. Não adianta nada
falar sobre como os tecidos de algodão são baratos e sobre seus efeitos
na civilização. Essas trabalhadoras honestas, no entanto, essas mulheres
absolutamente dignas de respeito, precisam chamar essa miséria de
``lar''.

Quando a refeição terminou, todas as mulheres se dirigiram para a
escrivaninha no canto, onde estava a sra. Stanard, e pagaram suas
contas. Eu recebi um bilhete vermelho esfarrapado, entregue por aquele
remendo de humanidade que fora minha garçonete. A conta somava cerca de
trinta centavos.

Depois da refeição, voltei ao térreo e reocupei meu lugar na sala dos
fundos. Estava com muito frio e desconfortável e decidi que não
conseguiria aguentar aquela situação por muito tempo, então quanto mais
cedo assumisse minha loucura, mais cedo me libertaria daquele ócio
forçado. Ah! Mas aquele foi o dia mais longo da minha vida. Fiquei
parada, assistindo as mulheres sentadas no salão da frente, onde todas,
exceto eu, estavam sentadas.

Uma não fazia nada além de ler, coçar a cabeça e ocasionalmente chamar
``Georgie!'' sem erguer os olhos do livro. ``Georgie'' era seu filho
traquinas, que fazia mais barulho e arruaça do que qualquer criança que
eu encontrara na vida. Ele fazia tudo que era grosseiro e mal-educado,
pensei, e a mãe nunca dizia nada a menos que outra pessoa gritasse com
ele. Outra mulher sempre pegava no sono e acordava com os próprios
roncos. Fiquei muito grata que ninguém mais acordava além dela mesma. A
maioria das mulheres ficava sentada sem fazer nada, mas algumas
tricotavam e arrendavam sem parar. A campainha gigante parecia tocar o
tempo inteiro, e com ela corria a menina de cabelos curtos. Esta, ainda
por cima, era uma dessas meninas que está sempre cantando pedaços de
todas as canções e hinos que foram compostos nos últimos cinquenta anos.
Ainda existe martírio nos tempos modernos. A campainha trazia mais
pessoas em busca de abrigo naquela noite. Exceto por uma mulher, que
viera do interior em uma viagem de compras, eram todas trabalhadoras,
algumas delas com filhos.

Quanto a noite se aproximou, a sra. Stanard veio me procurar.

--- O que há com você? Alguma tristeza, algum problema?

--- Não --- disse, quase chocada com a sugestão. --- Por quê?

--- Porque sim --- ela respondeu femininamente. --- Estou vendo no seu
rosto. Ele conta uma história difícil.

--- Sim, tudo é muito triste --- disse distraída, que era como pretendia
refletir minha loucura.

--- Mas não deixe isso lhe incomodar. Todos temos nossos problemas, mas
mais cedo ou mais tarde eles são superados. Que tipo de trabalho você
está procurando?

--- Não sei, é tudo muito triste --- respondi.

--- Você gostaria de ser governanta de crianças, usar uma boina branca e
um avental bem bonito? --- ela perguntou.

Ergui meu lenço até o rosto para esconder um sorriso. @--- Nunca
trabalhei, não sei como se faz --- respondi com a voz abafada.

--- Mas você precisa aprender --- ela insistiu. --- Todas essas mulheres
aqui trabalham.

--- É mesmo? --- sussurrei assombrada. --- Elas parecem horríveis para
mim, umas loucas. Tenho muito medo delas.

--- Elas não são muito simpáticas --- a outra concordou. --- Mas são
boas mulheres, são honestas e trabalhadoras. Não ficamos com as loucas
por aqui.

Mais uma vez usei meu lenço para esconder um sorriso, pensando que antes
do sol nascer, ela consideraria que havia ao menos uma louca entre o
rebanho.

--- Elas parecem todas loucas --- repeti. --- Estou com medo delas. O
mundo está cheio de gente louca, nunca se sabe o que eles vão fazer.
Matam tanta gente e a polícia nunca pega os assassinos. @ Terminei minha
apresentação com um soluço que teria comovido uma plateia de críticos
blasés. Ela deu um salto repentino, quase uma convulsão, e tive certeza
que meu primeiro golpe acertara em cheio. Foi divertido ver quão pouco
tempo ela demorou para se erguer da cadeira e sussurrar apressada: @---
Depois eu volto para conversar com você. @ Eu sabia que ela não
voltaria, e não voltou.

Quando a campainha da janta tocou, segui todas as outras até o porão e
fiz minha refeição noturna, muito parecida com o almoço, exceto pelo
fato do cardápio ser menor e de haver mais pessoas, pois as mulheres que
trabalhavam fora durante o dia estavam de volta. Depois do jantar, todas
nos retiramos para os salões, onde ficamos sentadas, ou então de pé, já
que não havia cadeiras o suficiente.

Foi uma noite horrivelmente solitária, e a luz que emanava do bico de
gás no salão e do lampião na entrada ajudava a nos mergulhar em penumbra
e manchava nossos espíritos de azul-marinho. Senti que não seria preciso
me sujeitar a muitas inundações da atmosfera para que ficasse pronta
para ser mandada ao lugar que estava trabalhando para entrar.

\imagemmedia{}{./media/rId30.png}Observei
duas mulheres, que pareciam ser as duas mais comunicativas do grupo, e
escolhi-as para concretizar minha salvação, ou, para ser mais exata,
minha condenação. Pedi licença e, declarando que estava solitária,
perguntei se poderia me juntar à conversa. Elas concordaram
graciosamente, então, ainda de chapéu e luvas, que ninguém me pedira
para retirar, sentei-me e escutei a conversa tediosa daquelas mulheres,
sem nunca participar, apenas tentando manter minha expressão triste e
respondendo suas observações com ``sim'' ou ``não'' ou ``não sei''.
Várias vezes, disse a elas que achava que todas naquela casa pareciam
malucas, mas elas demoraram a escutar minha frase peculiar. Uma disse
que seu nome era King e que vinha do sul, e então disse que meu sotaque
era sulista. Ela me perguntou bruscamente se eu não era do sul também.
Respondi que sim. A outra mulher começou a falar sobre os navios para
Boston e me perguntou se sabia a que horas eles partiam.

Por um instante, esqueci meu papel de insana e informei o horário
correto de partida. Em seguida, ela me perguntou que trabalho eu
pretendia fazer ou se já havia trabalhado antes. Respondi que achava
muito triste que havia tantos trabalhadores no mundo. Em resposta, ela
disse que fora infeliz e que viera para Nova Iorque, onde trabalhara
corrigindo as provas de um dicionário de medicina, mas que sua saúde
sofrera com a tarefa e que agora estava voltando para Boston. Quando a
criada veio nos dizer que era hora de dormir, afirmei que estava com
medo e repeti mais uma vez que todas as mulheres daquela casa pareciam
loucas. A ama insistiu que eu fosse me deitar. Perguntei se não poderia
ficar sentada nas escadas. @--- Não --- ela respondeu decidida. ---
Desse jeito, todo mundo ia achar que você é maluca. @ Finalmente,
permiti que me levassem para um quarto.

Neste momento, preciso apresentar por nome uma nova personagem da minha
narrativa: a mulher que fora revisora e que estava prestes a voltar para
Boston. Seu nome era sra. Caine, e ela era igualmente corajosa e
bondosa. A sra. Caine entrou em meu quarto, sentou-se e conversou comigo
por bastante tempo, soltando meu penteado com delicadeza. Ela tentou me
convencer a me despir e ir dormir, mas teimei em dizer que não. Durante
esse período, diversas hóspedes da casa foram se reunindo ao nosso
redor, expressando-se de diversas maneiras: @--- Pobre louca! @--- Mas é
maluca! @--- Estou com medo de ficar com uma maluca dessas aqui dentro.
@--- Ela vai matar todas nós de madrugada. @ Uma das mulheres defendia
que chamassem um policial para me levar imediatamente. Todas estavam
absolutamente aterrorizadas.

Ninguém queria se responsabilizar por mim e a mulher que deveria ocupar
o quarto comigo declarou que não ficaria com ``aquela maluca'' por todo
o dinheiro dos Vanderbilts. Foi então que a sra. Caine disse que ficaria
comigo. Eu disse a ela que gostaria disso, então ela foi deixada comigo.
Ela não se despiu, apenas deitou na cama, atenta a meus movimentos, e
tentou me convencer a me deitar. Eu estava com medo disso, no entanto,
pois sabia que se aceitasse, pegaria no sono e teria os sonhos pacíficos
e agradáveis de um bebê. Eu correria o risco, para usar uma gíria, de
``me dedurar''. Assim, insisti em ficar sentada na lateral da cama,
olhando fixamente para o nada. Minha pobre companheira acabou presa em
um estado de infelicidade terrível. De poucos em poucos minutos, ela se
erguia para olhar para mim. Ela disse que meus olhos tinham um brilho
terrível e começou a me fazer perguntas: onde eu morara, desde quando
estava em Nova Iorque e o que tinha feito, entre muitas outras coisas. A
todas essas perguntas, eu dava sempre a mesma resposta: dizia que
esquecera de tudo, que não conseguia lembrar de nada desde que a dor de
cabeça começara.

Pobre criatura! Como foi cruel o modo como a torturei, e quanta bondade
ela tinha no coração! Mas como torturei todas elas! Uma teve um pesadelo
comigo. Depois que havia passado uma hora no quarto, tomei um susto
quando escutei uma mulher gritando no quarto ao lado. Comecei a imaginar
que já estava em um hospício.

A sra. Caine acordou, olhou para os lados, assustada, e prestou atenção.
Ela saiu da cama e foi até o quarto ao lado. Eu conseguia escutar ela
fazendo perguntas para a outra mulher. Quando voltou, ela me contou que
a mulher havia tido um pesadelo horroroso. Ela sonhara comigo. Pelo que
a sra. Caine contou, ela me vira correndo com uma faca na mão para
matá-la. Tentando fugir de mim, ela tivera a sorte de conseguir gritar,
acordando a si mesma e espantando seu pesadelo. Depois disso, a sra.
Caine voltou para a cama, muito agitada, mas com muito sono.

Eu também estava cansada, mas tinha me preparado para trabalhar e estava
decidida a permanecer acordada a noite toda para efetuar meu fingimento
e atingir meu objetivo pela amanhã. Quando ouvi o toque da meia-noite,
pensei que ainda teria de esperar seis horas pelo nascer do sol. O tempo
passou com uma lentidão atroz. Os minutos pareciam horas. Os barulhos da
casa e da avenida morreram.

Com medo de que o sono seria convincente demais, comecei a analisar a
história de minha vida. Tudo parece tão estranho! Um incidente, mesmo
que nada trivial, é apenas mais um elo que nos acorrenta a um destino
imutável. Comecei pelo começo e repassei toda a minha história. Velhos
amigos foram relembrados com prazer; velhos inimigos, velhas dores no
coração, velhas alegrias se fizeram presentes mais uma vez. As páginas
viradas de minha vida foram desviradas e o passado se fez presente.

Com isso terminado, reuni minha coragem e comecei a pensar sobre o
futuro, imaginando, primeiro, o que aconteceria no dia seguinte, depois
fazendo planos sobre como executar meu projeto. Tentei imaginar se
conseguiria atravessar o rio e realizar minha estranha ambição,
tornando-me prisioneira das alas povoadas por minhas irmãs mentalmente
arruinadas. E depois de entrar, como seria minha experiência? E depois
disso? Como sair? @--- Bah! --- disse comigo mesma. --- Vão me tirar de
lá.

Foi a maior noite de minha existência. Por algumas horas, fiquei frente
a frente comigo mesma.

Olhei pela janela e recebi com alegria o primeiro raio trêmulo da
aurora. A luz foi ficando mais forte e cinzenta, mas o silêncio ainda
era incrivelmente forte. Minha companheira dormia. Eu ainda tinha uma
hora ou duas para passar. Felizmente, encontrei alguma utilidade para
minhas atividades mentais. No cativeiro, Robert Bruce ganhara confiança
no futuro e passara seu tempo, com todo o prazer possível sob suas
circunstâncias, assistindo a famosa aranha construir sua teia. Eu tinha
pestes menos nobres com as quais me interessar. Contudo, creio que fiz
algumas descobertas valiosas para a história natural. Estava prestes a
pegar no sono, apesar de toda a minha resistência, quando um susto me
deixou desperta. Achei que havia escutado algo se arrastar e cair sobre
o cobertor com um baque surdo.

Tive a oportunidade de estudar esses animais interessantes
minuciosamente. Claramente, eles haviam chegado em busca de café da
manhã e ficaram muito decepcionados ao descobrir que seu prato principal
não estava a sua espera. Eles chisparam pelo travesseiro, se reuniram,
pareceram realizar um diálogo deveras interessante e agiram, em todos os
aspectos, como se estivessem confusos pela ausência de um desjejum
apetitoso sobre a cama. Após uma última consulta prolongada, as
criaturas finalmente desapareceram em busca de outras vítimas. Só me
restou passar os longos minutos dando atenção às baratas, cujo tamanho e
agilidade me deixaram surpresa.

Minha companheira de quarto estava em sono profundo havia bastante
tempo, mas agora ela acordou e ficou surpresa em me ver ainda acordada e
aparentemente cheia de energia. Ela demonstrou a solidariedade de
sempre, aproximando-se de mim e segurando minhas mãos na tentativa de me
consolar, e então me perguntou se eu não queria voltar para casa. Ela me
manteve no quarto até quase todas as moradoras terem saído e então me
levou até o porão em busca de café e um pão. Depois dessa refeição
silenciosa, voltei para meu quarto, onde fiquei sentada, me lamentando.
A sra. Caine foi ficando cada vez mais ansiosa. @--- O que fazer? ---
ela exclamava. --- Onde estão seus amigos? @--- Não, eu não tenho amigos
--- respondi. --- Mas tenho malas. Onde elas estão? Eu quero as minhas
malas. @ A boa senhora tentou me acalmar, dizendo que as malas seriam
encontradas mais cedo ou mais tarde. Ela achava que eu era louca.

Mas eu a perdoo. É apenas depois que se está em apuros que se percebe
como a bondade e a solidariedade estão em falta neste mundo. As mulheres
no Lar que não tinham medo de mim quiseram se divertir às minhas custas,
então ficaram me incomodando com perguntas e observações que, se eu
realmente fosse louca, teriam sido cruéis e desumanas. Foi apenas essa
mulher, a bela e delicada sra. Caine, que demonstrou qualquer sentimento
feminino real. Ela fez com que as outras parassem de me provocar e
trocou de cama com a mulher que se recusava a dormir ao meu lado. Ela
protestou contra a sugestão de me deixar sozinha e de me trancar a noite
toda para que não pudesse machucar ninguém, insistiu em permanecer
comigo para prestar qualquer auxílio que viesse a ser necessário,
penteou meu cabelo, banhou meu cenho e falou comigo no tom que uma mãe
usaria com uma filha doente. Usando de todos os meios possíveis, ela
tentou me fazer ir para a cama e descansar, e quando chegou a manhã ela
se ergueu e colocou um cobertor a minha volta, com medo que eu passasse
frio. Finalmente, ela me deu um beijo na testa e sussurrou, cheia de
compaixão:

--- Pobrezinha! Pobrezinha!

Como admirei a bondade e a coragem daquela mulherzinha. Como quis
reconfortá-la e sussurrar que não era louca. E como torci para que, se
alguma pobre menina tivesse a infelicidade de ser o que eu apenas
fingia, ela encontrasse uma pessoa dona do mesmo espírito humanitário
que a sra. Ruth Caine.



\chapter{O juiz duffy e a
polícia}\label{capuxedtulo-iv-o-juiz-duffy-e-a-poluxedcia}

Mas voltemos à minha história. Continuei interpretando meu papel até a
Sra. Stanard, a assistente da governanta, aparecer. Ela tentou me
convencer a permanecer calma. Comecei a perceber que ela me queria tirar
da casa a todo custo, mas discretamente, se possível. Isso eu não
queria. Não saí do lugar, repetindo sempre meu refrão sobre as malas
perdidas. Finalmente, alguém sugeriu que um policial fosse chamado.
Depois de algum tempo, a sra. Stanard vestiu seu toucado e saiu. Com
isso, eu soube que estava avançando em direção ao lar de insanidade. A
mulher não demorou a voltar, trazendo consigo dois policiais, homens
altos e fortes, que entraram no quarto sem cerimônia, claramente
esperando encontrar uma louca violenta. O nome de um deles era Tom
Bockert.

\imagemmedia{}{./media/rId33.png}Quando
entraram, fingi que não os vi. @--- Quero que vocês levem ela embora sem
escândalo --- a sra. Stanard disse. @--- Se ela não vier quieta, vou
arrastá-la pela rua --- um dos homens respondeu. @ Continuei sem dar
atenção e eles, mas com certeza não queria criar um escândalo fora da
casa. Felizmente, a sra. Caine apareceu para me resgatar. Ela contou aos
policiais que eu havia reclamado sobre malas perdidas, e juntos eles
criaram um plano para fazer com que fosse com eles calmamente, dizendo
que iriam comigo procurar meus pertences. Eles me perguntaram se iria.
Respondi que estava com medo de ir sozinha. A sra. Stanard disse que me
acompanharia e combinou com os dois policiais para que nos seguissem a
uma distância respeitosa. Ela amarrou meu véu e nós saímos da casa pelo
porão, seguindo para o outro lado da cidade com os policiais atrás de
nós. Fomos caminhando em silêncio até chegarmos à delegacia, que aquela
boa senhora me garantiu ser um escritório de entregas onde certamente
encontraríamos meus pertences. Entrei no edifício tremendo de medo, e
com bom motivo.

Alguns dias antes disso, eu conhecera o Capitão McCullagh em uma reunião
na Cooper Union. Naquela ocasião, eu havia pedido algumas informações.
Se estivesse na delegacia, ele me reconheceria? Se sim, a ideia de
entrar no hospício da ilha estaria perdida. Baixei meu chapéu de
marinheiro tudo o que pude sobre o rosto e me preparei para a provação.
Para o meu azar, o corpulento Capitão McCullagh estava na entrada, junto
à escrivaninha.

Ele me observou atentamente enquanto o oficial na escrivaninha
conversava baixinho com a sra. Stanard e o policial que me trouxera.

--- Você é Nellie Brown? --- perguntou o oficial. @ Respondi que sim.
@--- De onde você veio? @ Respondi que não sabia, mas então a sra.
Stanard forneceu diversas informações sobre mim. Ela contou sobre meus
modos estranhos em seu lar, sobre como passara a noite inteira acordada
e que, na sua opinião, eu era uma pobre infeliz enlouquecida pelo
tratamento desumano. Depois de uma conversa entre a sra. Stanard e os
dois policiais, Tom Bockert recebeu a ordem de nos levar até o tribunal
em um carro.

--- Pode vir --- Bockert disse. --- Vou achar suas malas para você. @
Fomos todos juntos, a sra. Stanard, Tom Bockert e eu. Eu disse que era
muita bondade deles me acompanhar e que não esqueceria disso tão cedo.
Enquanto caminhávamos, fui repetindo meu refrão sobre as malas,
misturado com observações ocasionais sobre a sujeira das ruas e a
aparência curiosa das pessoas que encontrávamos no caminho. @--- Acho
que nunca vi gente assim na vida. Quem são? --- perguntei. @ Meus
companheiros olharam para mim com expressões de piedade. Evidentemente,
eles achavam que eu era uma estrangeira, uma emigrante ou algo do tipo.
Eles disseram que as pessoas ao meu redor eram trabalhadores. Observei
mais uma vez que achava que havia trabalhadores demais no mundo para a
quantidade de trabalho que havia para ser feita. Depois dessa frase, o
oficial P. T. Bockert olhou para mim com muita atenção. Claramente, ele
achava que minha mente não tinha salvação. Passamos por vários outros
policiais, a maioria dos quais perguntou a meus guardiões o que havia de
errado comigo. A essa altura, um certo número de crianças maltrapilhas
havia começado a nos seguir, fazendo observações sobre mim que eram
originais e divertidas.

--- Pegaram ela por quê? @--- Ei, tira, onde cês pegaram ela? @--- Onde
é que cês acharam ela? @--- Mas que belezura!

A pobre sra. Stanard ficou mais amedrontada do que eu. A situação foi
ficando interessante, mas eu ainda temia meu encontro com o juiz.

Finalmente, chegamos a um prédio baixo, onde Tom Bockert teve a bondade
me informar: @--- Este é o escritório de entregas. Agora vamos encontrar
essas suas malas.

A entrada do edifício estava cercada por um grupo de curiosos, mas como
não acreditava que meu caso fosse tão ruim que pudesse passar por eles
sem alguma observação, perguntei se toda aquela gente havia perdido suas
malas também.

--- Sim --- ele respondeu. --- Quase todos eles estão procurando suas
malas.

--- Eles também parecem ser todos estrangeiros. @--- Sim --- Tom
respondeu. --- São todos estrangeiros que acabam de chegar. Eles
perderam suas malas e nós gastamos quase todo o nosso tempo ajudando a
achá-las.

Entramos no tribunal, o Tribunal de Polícia de Essex Market. A questão
da minha sanidade ou insanidade finalmente seria decidida. O Juiz Duffy
estava sentado atrás de uma escrivaninha alta, com um olhar que parecia
sugerir que ele distribuía a essência da bondade humana por atacado. Eu
temia não ter o destino que buscava por causa da bondade que via em cada
ruga e cada linha de seu rosto. Eu estava cada vez mais ansiosa. Depois
que Tom Bockert fez sua apresentação sobre o caso, segui a sra. Stanard
quando ela foi convocada para se apresentar.

--- Venha cá --- disse um oficial. --- Qual é o seu nome?

--- Nellie Brown --- respondi com um sotaque discreto. --- Perdi minhas
malas e gostaria que você as encontrasse.

--- Quando você veio para Nova Iorque? --- ele perguntou.

--- Eu não vim para Nova Iorque --- respondi (adicionando mentalmente:
``porque estou aqui há algum tempo''.)

--- Mas você está em Nova Iorque --- o homem disse.

--- Não --- eu disse, parecendo tão incrédula quanto imaginava que uma
louca pareceria. --- Eu não vim para Nova Iorque.

--- Essa menina veio do oeste --- ele disse em um tom que me fez tremer.
--- Ela tem um sotaque do oeste.

Um dos ouvintes desse breve diálogo afirmou que havia morado no sul e
que meu sotaque era sulista, enquanto outro oficial tinha certeza que eu
vinha do leste. Fiquei aliviada quando o primeiro se virou para o juiz e
disse:

--- Senhor juiz, estamos perante o caso peculiar de uma jovem que não
sabe quem é ou de veio. É melhor que o senhor resolva o caso de
imediato.

\imagemmedia{}{./media/rId34.png}Comecei
a tremer, e não só de frio, então olhei para os lados. Havia uma
multidão estranha ao meu redor, composta de homens e mulheres mal
vestidos, e em seus rostos estavam estampadas histórias de vidas
difíceis, abuso e pobreza. Alguns falavam alegremente com os amigos,
enquanto outros estavam sentados com um olhar de desesperança absoluta.
O tribunal também tinha alguns oficiais bem-vestidos e bem alimentados
que assistiam a cena passivamente, quase indiferentes. Para eles, era
uma história antiga. Mais uma infeliz estava sendo adicionada a um longo
rol, uma lista que há muito tempo deixara de interessar ou preocupar
qualquer um deles.

--- Venha cá, menina, e levante esse véu --- o Juiz Duffy chamou, usando
um tom que me surpreendeu; não imaginei que aquele rosto bondoso fosse
capaz de ser tão áspero.

--- Com quem o senhor está falando? --- perguntei com toda a altivez.

--- Venha cá, minha cara, e levante o véu. Se a Rainha da Inglaterra
estivesse aqui, ela também precisaria erguer seu véu --- ele disse,
muito mais bondoso.

--- Assim está muito melhor --- respondi. --- Não sou a Rainha da
Inglaterra, mas vou erguer meu véu.

Quando o fiz, o pequeno juiz olhou para mim e, usando um tom muito
gentil e caridoso, perguntou:

--- Minha filha, o que houve?

--- Não houve nada, eu só perdi minhas malas. Este homem prometeu me
levar até onde elas estavam --- eu disse, indicando o policial Bockert.

--- O que você sabe sobre esta menina? --- o juiz perguntou em um tom
grave para a sra. Stanard, que estava ao meu lado, pálida e trêmula.

--- Não sei de nada, exceto que ela veio ao lar ontem e pediu para se
hospedar por uma noite.

--- O lar! O que você quer dizer por ``o lar''? --- o Juiz Duffy
perguntou rapidamente.

--- É um lar temporário para mulheres trabalhadoras que fica no número
84 da Segunda Avenida.

--- Qual é o seu cargo lá?

--- Eu sou assistente da governanta.

--- Muito bem, conte o que sabe sobre o caso.

--- Quando cheguei ao lar ontem, notei ela descendo pela avenida,
sozinha. Eu acabara de entrar quando a campainha soou e ela apareceu.
Quando conversamos, ela queria saber se poderia passar a noite, e eu
disse que sim. Depois de um tempo, ela disse que todas as pessoas dentro
da casa pareciam malucas e que tinha medo delas. Depois, começou a se
recusar a ir para a cama. Passou a noite inteira acordada.

--- Ela tinha dinheiro?

--- Sim --- respondi no lugar dela. --- Eu paguei tudo, e nunca comi
nada tão ruim na vida.

Essa frase provocou um sorriso geral, além de um murmúrio de ``ela não é
maluca nessa coisa da comida''.

--- Pobrezinha --- o Juiz Duffy disse. --- Ela está bem vestida e é uma
dama. Seu inglês é perfeito, eu apostaria qualquer coisa que é uma boa
menina. Tenho certeza que é a queridinha de alguém.

Essa declaração provocou uma risada geral. Eu mesma precisei cobrir o
rosto com um lenço e me esforcei para segurar a gargalhada que ameaçava
estragar meus planos, apesar de minhas decisões.

--- Quis dizer que ela é a filha querida de alguma mãe --- o juiz
consertou sem demora. --- Alguém deve estar procurando por ela.
Pobrezinha, vou ser bom para ela, pois ela se parece com a minha
falecida irmã.

Houve um silêncio momentâneo e os oficiais me lançaram um olhar mais
caridoso. Em silêncio, abençoei o bom coração daquele juiz, desejando
que todas as pobres criaturas afligidas pelo mal que eu fingia ter
tivessem que lidar com um homem tão bondoso quanto era o Juiz Duffy.

--- Eu queria que os repórteres estivessem aqui --- ele disse afinal.
--- Eles descobririam alguma coisa sobre a menina.

Fiquei aterrorizada com a ideia, pois se existe alguém capaz de
desvendar um mistério, é um repórter. Achei que seria melhor enfrentar
uma multidão de médicos especialistas, policiais e detetives do que dois
espécimes brilhantes de minha profissão. Foi por isso que disse:

--- Não entendo por que isso tudo é necessário para vocês me ajudarem a
achar minhas malas. Esses homens são atrevidos e não quero que fiquem me
encarando. Vou embora, não quero ficar aqui.

Com, isso baixei meu véu e torci em segredo para que os repórteres se
atrasassem por qualquer motivo até eu ser enviada ao hospício.

--- Não sei o que fazer com essa pobre menina --- o juiz disse,
preocupado. --- É preciso que tomem conta dela.

--- Mande-a para a ilha --- sugeriu um dos oficiais.

--- Não! --- a sra. Stanard disse, claramente alarmada. --- Não! Ela é
uma dama. Vai morrer se for mandada para a ilha.

Tive vontade de sacudir aquela boa senhora. Imagine, a ilha era
exatamente o local que desejava alcançar, e ali estava ela tentando me
impedir de ser mandada para lá! Era muita bondade dela, mas naquelas
condições, era enervante.

--- Algo de errado aconteceu aqui --- o juiz afirmou. --- Creio que a
menina foi drogada e trazida para esta cidade. Prepare os documentos,
ela será enviada a Bellevue para ser examinada. Em alguns dias, o efeito
da droga provavelmente passará e ela será capaz de contar uma história
chocante. Se ao menos os repórteres estivessem aqui!

Eu os temia, então disse algo sobre não querer mais ser objeto de todos
aqueles olhares. O Juiz Duffy disse a Bockert que eu deveria ser levada
ao escritório dos fundos. Quando nos acomodamos, o Juiz Duffy entrou e
me perguntou se meu lar ficava em Cuba.

--- Sim --- respondi com um sorriso. --- Como você sabia?

--- Ah, eu sabia, minha cara. Diga-me, onde era? Em que parte de Cuba?

--- Na \#\#hacienda\textbackslash{}\#\# --- respondi.

--- Ah! --- disse o juiz. --- Em uma fazenda. Você lembra de Havana?

--- \#\#Sí, señor\textbackslash{}\#\# --- respondi. --- Fica perto da
minha casa. Como você sabia?

--- Oh, sim, conheço muito bem. Quem sabe você me diz o nome da sua
casa? --- ele perguntou, persuasivo.

--- É isso que esqueço --- respondi tristonha. --- Tenho uma dor de
cabeça o tempo inteiro, ela faz com que eu me esqueça de coisas. Não
quero que me incomodem. Todo mundo fica me fazendo perguntas, isso faz
minha cabeça piorar. @ E era verdade.

--- Ninguém mais vai incomodá-la. Sente-se aqui e descanse --- o juiz
disse, simpático, e me deixou sozinha com a sra. Stanard.

Foi então que um oficial entrou junto com um repórter. Fiquei muito
amedrontada, temendo que seria reconhecida como jornalista. @--- Não
quero ver repórter nenhum --- eu disse, virando a cabeça para o outro
lado. --- Não vou ver ninguém, o juiz disse que não era para me
incomodarem.

--- Nada de insano nisso --- disse o homem que trouxera o repórter, e os
dois saíram da sala juntos. @ Mais uma vez, tive um ataque de medo.
Recusar o repórter teria sido longe demais? Minha sanidade fora
detectada? Se eu dera a impressão de ser sã, estava determinada a
desfazê-la, então dei um salto e comecei a correr ao redor do
escritório. A sra. Stanard se agarrou ao meu braço, aterrorizada.

--- Não vou ficar aqui. Eu quero minhas malas! Por que tanta gente me
incomoda? @ Continuei com isso até o cirurgião de ambulância aparecer,
acompanhado pelo juiz.

\label{section-2}

\chapter{Declarada
insana}\label{capuxedtulo-v-declarada-insana}

--- Temos no tribunal uma pobre menina que foi drogada --- o juiz
explicou. --- Ela se parece com a minha irmã e qualquer um enxerga que é
boazinha. Estou interessado na menina e faria por ela tanto quanto se
fosse minha própria filha. Quero que seja gentil com ela --- ele disse
para o cirurgião de ambulância. @ A seguir, voltando-se para a sra.
Stanard, ele perguntou se eu não poderia permanecer com ela por alguns
dias, até meu caso ser investigado. Felizmente, ela respondeu que não
poderia, pois todas as mulheres do Lar tinham medo de mim e iriam embora
se eu fosse mantida lá. Tive muito medo que ela permaneceria comigo se o
pagamento fosse garantido, então disse algo sobre a má qualidade da
comida e que não pretendia voltar para o Lar. A seguir, começou meu
exame. O médico parecia inteligente e eu não tinha esperança alguma de
enganá-lo, mas estava resolvida a continuar com a farsa.

--- Ponha a língua para fora --- ele ordenou com força.

Ri comigo mesma com a ideia.

--- Ponha a língua para fora quando eu mandar --- ele disse.

--- Não quero --- respondi honestamente.

--- Mas precisa. Você está doente e eu sou um doutor.

--- Não estou doente nem nunca estive. Só quero minhas malas.

\imagemmedia{}{./media/rId37.png}Mas
coloquei a língua para fora, que ele observou judiciosamente. A seguir,
ele mediu meu pulso e escutou os batimentos do meu coração. Eu não fazia
ideia de como bate o coração de uma pessoa insana, então prendi a
respiração enquanto ele escutava até que, quando o exame terminou,
fiquei ofegante para me recuperar. A seguir, ele experimentou o efeito
da luz sobre as pupilas de meus olhos. Erguendo a mão a um centímetro do
meu rosto, me mandou olhar para ela e então puxou-a bruscamente e
examinou meus olhos. Também estava perplexa sobre como a insanidade
aparecia nos olhos, então decidi que, sob essas circunstâncias, o melhor
seria encarar. Foi o que fiz. Mantive meus olhos cravados naquela mão e,
quando ele a removeu, me esforcei o quanto pude para não deixar que meus
olhos piscassem.

--- Que drogas você está tomando? --- ele me perguntou.

--- Drogas! --- repeti, admirada. --- Eu não sei o que são drogas.

--- As pupilas dos olhos dela estão dilatadas desde que chegou ao Lar.
Não mudaram nenhuma vez --- a sra. Stanard explicou; tentei imaginar
como ela sabia se minhas pupilas haviam mudado ou não, mas permaneci em
silêncio.

--- Creio que ela tem usado beladona --- o doutor afirmou, e pela
primeira vez fiquei grata de ser um pouco míope, o que obviamente
explica a dilatação de minhas pupilas. @ Considerei que poderia contar a
verdade quando isso não fosse prejudicar meu caso, então informei o
médico que era míope, que não estava doente, que nunca estivera doente e
que ninguém tinha o direito de me prender enquanto queria encontrar
minhas malas. Eu queria ir para casa. Ele escreveu por muito tempo em um
livro fino e comprido e então disse que ia me levar para casa. O juiz
disse para me levarem e serem bons comigo, para dizerem o mesmo à equipe
do hospital e também pedir que fizessem todo o possível por mim. Se ao
menos tivéssemos mais homens como o Juiz Duffy, as pobres infelizes não
passariam toda a vida nas trevas.

Comecei a ter mais confiança em minhas habilidades, já que agora um
juiz, um médico e toda uma multidão haviam me declarada insana, então
fiquei muito alegre em colocar meu véu quando fui informada que seria
levada de carruagem e que depois poderia ir para casa. @--- Fico muito
contente em ir com você --- eu disse, e era verdade. Estava contente
mesmo. @ Mais uma vez protegida por Bockert, atravessei o pequeno
tribunal lotado. Estava muito orgulhosa de mim mesma quando saí pela
porta lateral e entrei no beco onde a ambulância estava à espera.
Próximo aos portões trancados, havia um escritoriozinho ocupado por
diversos homens e por livros enormes. Todos entramos ali e, quando
começaram a me fazer perguntas, o médico interrompeu a conversa e disse
que tinha todos os documentos e que não adiantaria nada continuar me
questionando, pois eu não saberia responder. Foi um grande alívio, pois
meus nervos já estavam à flor da pele. Um homem grosseiro quis me
colocar na ambulância, mas recusei sua ajuda com tanta firmeza que o
médico e o policial ordenaram que ele desistisse e fizeram essa cortesia
eles mesmos. Não entrei na ambulância sem reclamar. Observei que nunca
havia visto aquele modelo de carruagem na vida e que não queria viajar
nela, mas depois de algum tempo deixei que me convencessem do contrário,
como sempre fora minha intenção.

Nunca vou esquecer aquela viagem. Depois de ser colocada sobre o
cobertor amarelo, o doutor entrou e sentou-se junto à porta. Os grandes
portões se abriram e a multidão de curiosos que se reunira abriu caminho
para a ambulância, que foi saindo de ré. Ah, como tentaram espiar a
suposta maluca! O médico viu que eu não queria ser um espetáculo para
aquela gente e teve a consideração de fechar as cortinas, sem esquecer
de consultar meus desejos sobre isso. Mesmo assim, as pessoas não se
afastaram. As crianças foram correndo atrás de nós, gritando uma
infinidade de gírias e tentando olhar por baixo das cortinas. Foi um
passeio muito interessante, mas também extremamente doloroso. Eu me
agarrei, ainda que não houvesse muito ao qual se agarrar, e o cocheiro
foi dirigindo como se temesse que alguém iria nos alcançar.

\label{section-3}

\chapter{No hospital de
bellevue}\label{capuxedtulo-vi-no-hospital-de-bellevue}

Finalmente chegamos a Bellevue, minha terceira parada a caminho da ilha.
Eu fora bem-sucedida em minhas provações no lar e no Tribunal de Polícia
de Essex Market e agora estava confiante no futuro. A ambulância parou
com uma freada súbita e o doutor saltou para fora. @--- Quantas você
trouxe? --- ouvi alguém perguntar. @--- Apenas uma, para o pavilhão ---
foi a resposta. @ Um homem de aparência grosseira apareceu e me agarrou,
tentando me arrastar para fora como se eu tivesse a força de um elefante
e estivesse disposta a resistir. O médico, ao ver minha reação enojada,
ordenou que o homem me deixasse em paz, dizendo que ele próprio cuidaria
de mim. Com isso, ele me ergueu com cuidado para fora do coche e eu pude
caminhar com toda a graciosidade de uma rainha, passando pela multidão
de curiosos que se reunira para ver a chegada da nova infeliz. Junto com
o médico, entrei em um escritoriozinho escuro onde vários homens estavam
trabalhando. Um deles, atrás da escrivaninha, abriu um livro e começou a
fazer a mesma longa série de perguntas que eu já havia encontrado tantas
vezes.

Recusei-me a responder e o médico disse que não seria necessário me
incomodar mais, pois todos os documentos estavam preenchidos e eu era
insana demais para fornecer qualquer informação relevante. Fiquei
aliviada com a facilidade que encontrar. Apesar de ainda decidida,
começara a me sentir tonta devido à falta de comida. Nesse momento, foi
dada a ordem para que eu fosse levada ao pavilhão dos insanos, então um
homem musculoso apareceu e me agarrou tão forte pelo braço que uma dor
atravessou todo o meu corpo. Fiquei furiosa, e por um instante esqueci
meu papel.

--- Como você ousa encostar em mim?! --- gritei, virando-me para ele. @
Em resposta, ele me soltou um pouco, e eu o afastei com mais força do
que sabia ter.

--- Não irei com ninguém além deste homem --- eu disse, apontando para o
cirurgião de ambulância. --- O juiz disse que ele cuidaria de mim, não
vou ir com mais ninguém.

Em resposta, o cirurgião disse que me levaria. Saímos de braços dados,
seguindo o homem que fora tão grosseiro comigo. Atravessamos o jardim
bem cuidado e finalmente chegamos à ala dos insanos. Uma enfermeira de
boina branca estava preparada para me receber.

--- Esta menina está aqui à espera do barco --- o cirurgião disse, e
começou a ir embora. @ Eu implorei que ele não fosse, ou que me levasse
consigo, mas ele respondeu que queria almoçar antes e que eu deveria
esperar por ele. Quando insisti em acompanhá-lo, ele alegou que
precisava auxiliar em uma amputação e que não seria bom para mim estar
presente nessa situação. Era evidente que ele acreditava estar lidando
com uma pessoa insana. Nesse momento, os gritos de loucura mais
horríveis ecoaram do jardim nos fundos. Apesar de toda a minha bravura,
senti um calafrio com a ideia de ser presa com uma criatura que
realmente era louca. O doutor evidentemente percebeu meu nervosismo,
pois se voltou para a atendente e disse:

--- Como os carpinteiros são barulhentos.

Virando-se para mim, ele ofereceu a seguinte explicação: novos edifícios
estavam sendo construídos e o ruído vinha de alguns dos operários
envolvidos no projeto. Eu disse que não queria ficar ali sem ele e, para
me apaziguar, ele prometeu voltar em breve. O médico foi embora e eu
finalmente me tornei moradora de um hospício.

Parei junto à porta e contemplei a cena a minha frente. O longo
corredor, sem tapetes ou carpetes, tinha sido escovado até adquirir
aquela brancura peculiar encontrada apenas em instituições públicas. Nos
fundos, grandes portas de ferro estavam trancadas por um cadeado.
Diversos bancos rígidos e cadeiras de vime representam o único
mobiliário do local. Em ambos os lados do corredor, portas levavam ao
que eu imaginava ser, e descobri serem, quartos. À direita da porta de
entrada havia uma pequena sala de estar para as enfermeiras, enquanto no
outro lado estava a sala onde as refeições eram distribuídas. Uma
enfermeira de vestido preto, boina branca e avental, armada com um molho
de chaves, estava encarregada do lugar. Logo descobri que seu nome era
Srta. Ball.

Uma senhora irlandesa era a criada de serviços gerais. Eu ouvi ela ser
chamada de Mary e fico feliz em saber que aquele lugar inclui uma mulher
de tão bom coração. Da parte dela, recebi apenas bondade e a mais
absoluta consideração. Havia apenas três pacientes, como são chamadas,
no local. Eu seria a quarta. Decidi que podia começar meu trabalho
imediatamente, pois ainda esperava que o primeiro doutor que encontrasse
me declararia sã e me mandaria de volta para o mundo. Assim, dirigi-me
ao fundo da sala, me apresentei para uma das mulheres e pedi que ela
contasse sua história. Seu nome, disse ela, era Srta. Anne Neville, e
ela adoecera por excesso de trabalho. Ela trabalhava como camareira, mas
quando sua saúde se deteriorou, ela foi enviada ao Lar das Irmãs para
ser tratada. Seu sobrinho, que era garçom, estava desempregado e, por
não ser capaz de pagar suas despesas no Lar, arranjou sua transferência
para Bellevue.

--- Você também sofre de algo mentalmente? --- perguntei.

--- Não. Os médicos andam me fazendo muitas perguntas esquisitas e me
confundem o tempo inteiro, mas não tem nada de errado com o meu cérebro.

--- Você sabia que apenas pessoas insanas são mandadas para este
pavilhão? --- perguntei.

--- Sim, eu sei, mas não consigo fazer nada. Os médicos se recusam a me
escutar, e não adianta dizer nada para as enfermeiras.

Satisfeita por diversos motivos que a Srta. Neville era tão sã quanto eu
própria, transferi minhas atenções para uma das outras pacientes. Ela
precisava de cuidados médicos e era mentalmente apatetada, ainda que já
tenha visto muitas mulheres das classes mais baixas, cuja sanidade
jamais é questionada, que não demonstravam maior inteligência.

A terceira paciente, a sra. Fox, não dizia muito. Ela era bastante
quieta e, após me informar que seu caso não tinha salvação, se recusou a
falar. Comecei a ter mais certeza de minha posição e conclui que médico
nenhum me convenceria de minha sanidade enquanto tivesse alguma
esperança de completar minha missão. Uma enfermeira baixinha, de pele
alva, chegou e, após colocar sua boina, disse à sra. Ball que deveria ir
almoçar. A nova enfermeira, chamada de Srta. Scott, se aproximou de mim
e disse com aspereza:

--- Tire o chapéu.

--- Não vou tirar meu chapéu --- respondi. --- Estou esperando pelo
barco, não vou removê-lo.

--- Você não vai ir em barco nenhum, tanto faz saber disso agora ou
depois. Você está em um hospício.

Apesar de estar absolutamente ciente do fato, ouvir aquelas palavras sem
enfeites ou rodeios me deu um choque. @--- Não queria vir para cá, não
estou doente, não sou louca. Não vou ficar.

--- Vai demorar muito para sair daqui se não obedecer --- a Srta. Scott
respondeu. --- Pode ir tirando o chapéu, ou vou ter que usar a força. Se
não conseguir, só preciso tocar uma campainha para ter ajuda. Agora você
vai tirar?

--- Não, não vou. Estou com frio, quero ficar de chapéu e você não pode
me obrigar a tirá-lo.

--- Vou lhe dar mais alguns minutos. Se não tirá-lo sozinha, vai ter que
ser a força. E vou avisando, não será nada delicado.

--- Se tirar meu chapéu, vou tirar sua boina. E então?

A Srta. Scott foi chamada para a porta nesse instante e fiquei com medo
que uma demonstração de compostura poderia indicar sanidade demais,
então tirei meu chapéu e luvas e fiquei sentada em silêncio, olhando
para o nada até ela voltar. Eu estava com fome e fiquei muito satisfeita
em ver que Mary estava começando a preparação para o almoço. O trabalho
era simples: ela apenas puxou um banco reto junto a uma mesa sem toalha
e ordenou que os pacientes se reunissem em torno do banquete; a seguir,
ela trouxe um pratinho de lata com um pedaço de carne cozida e uma
batata. A comida não estaria mais fria se tivesse sido cozinhada na
semana anterior e jamais tivera a chance de encontrar sal ou pimenta. Eu
não quis ir até a mesa, então Mary se dirigiu ao canto onde estava
sentada. Enquanto distribuía comida do pratinha de lata, ela me
perguntou:

--- Você tem alguma moedinha, querida?

--- O quê? --- eu disse, surpresa.

--- Você tem alguma moedinha, querida, que poderia me dar? Vão tirar de
você de qualquer jeito, querida, então não faz diferença dar elas para
mim.

Agora entendi exatamente a situação, mas não pretendia recompensar Mary
logo no início, temendo que isso influenciaria o tratamento que ela me
daria. Respondi que perdera minha bolsa, o que era verdade. Apesar de
não ter dado a Mary dinheiro algum, ela não foi menos bondosa comigo.
Quando reclamei do prato de lata no qual trouxera minha comida, ela
buscou um de porcelana para mim. Quando foi impossível consumir a comida
que apresentara, ela me trouxe um copo de leite e uma bolacha de água e
sal.

Todas as janelas da sala estavam abertas e o ar frio começou a afetar
meu sangue sulista. Quando o frio estava quase insuportável, reclamei
para a Srta. Scott e a Srta. Ball, mas sua resposta brusca foi que eu
estava em uma instituição de caridade e não devia esperar demais. Todas
as outras mulheres sofriam do frio e as próprias enfermeiras precisavam
vestir roupas pesadas para se aquecerem. Perguntei se poderia me deitar.
@--- Não! --- elas responderam. @ Finalmente, a Srta. Scott buscou um
velho xale cinza e, depois de espantar as traças, mandou que eu o
colocasse.

--- Esse xale não parece muito bom --- respondi.

--- Bem, algumas pessoas se dariam melhor se não fossem tão orgulhosas
--- a Srta. Scott disse. --- Quem depende de caridade não deveria
esperar nada e não pode reclamar.

Coloquei o xale, roído por traças e bolorento como era, ao meu redor e
sentei-me em uma cadeira de vime, tentando imaginar o que aconteceria a
seguir, se morreria de frio ou sobreviveria. Meu nariz estava gelado,
então cobri metade da cabeça e quase peguei no sono, mas, de repente, o
xale foi puxado do meu rosto. A Srta. Scott estava à minha frente ao
lado de um homem estranho, que depois se revelou um médico.

--- Já vi esse rosto antes --- foi o modo como me cumprimentou.

--- Então você me conhece? --- perguntei, demonstrando uma animação que
não sentia.

--- Creio que sim. De onde você vem?

--- De casa.

--- Onde fica sua casa?

--- Você não sabe? Em Cuba.

\imagemmedia{}{./media/rId40.png}Ele
sentou-se ao meu lado, mediu meu pulso e examinou minha língua.

--- Conte à Srta. Scott sua história --- ele pediu afinal.

--- Não, não vou falar com mulheres.

--- O que você faz em Nova Iorque?

--- Nada.

--- Você pode trabalhar?

--- Não, \#\#señor\textbackslash{}\#\#.

--- Diga-me, você é uma mulher das ruas?

--- Não estou entendendo --- respondi, nauseada dele.

--- Quero dizer, você permitiu que homens lhe sustentassem?

Senti vontade de lhe dar uma bofetada, mas precisava manter minha
compostura, então respondi simplesmente:

--- Não sei do que você está falando. Sempre morei em casa.

Depois de muitas outras perguntas, todas igualmente inúteis e
disparatadas, ele me deixou e foi conversar com a enfermeira. @---
Absolutamente demente --- ele disse. --- Não vejo nenhuma esperança para
o caso. Ela precisa ser colocada em algum lugar onde alguém tomará conta
dela.

E assim passei pelo meu segundo especialista médico.

Depois disso, comecei a ter menos apreço do que nunca pela capacidade
dos médicos, e mais pela minha própria. Agora tinha certeza de que
médico nenhum tinha como saber se as pessoas eram insanas ou não, desde
que o caso não fosse violento.

Durante a tarde, um menino e uma mulher apareceram. A mulher sentou-se
em um banco, enquanto o menino entrou e conversou com a Srta. Scott.
Logo em seguida, ele saiu, acenou adeus para a mulher, que era sua mãe,
e foi embora. Ela não parecia insana, mas como era alemã, não pude
descobrir sua história. Seu nome, entretanto, era Louise Schanz. Ela
parecia perdida, mas quando as enfermeiras lhe colocaram para costurar,
a mulher trabalhou bem e com rapidez. Às três da tarde, todas as
pacientes receberam um caldo, e às cinco uma xícara de chá e um pedaço
de pão. Eu recebi tratamento especial: quando viram que me era
impossível comer o pão ou beber o líquido ao qual davam o nome de chá,
elas me deram um copo de leite e uma bolacha, assim como haviam feito ao
meio-dia.

Quando o gás estava sendo aceso, outra paciente foi adicionada ao grupo,
uma menina jovem, de vinte e cinco anos. Ela me disse que estava acamada
até então, e sua aparência confirmava a história. Ela parecia alguém que
sofrera de um ataque de febre gravíssimo. @--- Agora estou sofrendo de
uma debilidade nervosa, meus amigos me enviaram aqui para ser tratada. @
Não contei a ela onde estava, e ela parecia bastante satisfeita. Às seis
e quinze, a Srta. Ball disse que queria ir embora, então todas
precisaríamos nos deitar. Cada uma de nós, que agora éramos seis, foi
colocada em um quarto e precisou se despir. No lugar, recebi uma
camisola curta de flanela de algodão para usar durante a noite. Depois
disso, a enfermeira recolheu todos os itens que eu vestira durante o
dia, fez uma trouxa na qual anotou ``Brown'' e saiu. A janela tinha
grades de ferro e estava trancada. A Srta. Ball, depois de me fornecer
um cobertor adicional, que, segundo ela, era um favor raramente
concedido na instituição, foi embora e me deixou sozinha. A cama não era
confortável. De tão rígida, eu sequer conseguia amassá-la. O travesseiro
era recheado de palha. Sob os lençóis, encontrei uma colcha de oleado. A
noite foi esfriando, então tentei aquecer o oleado. Tentei e tentei, mas
quando amanheceu, a cama ainda estava tão fria quanto no instante em que
me deitei, e ainda havia me reduzido à temperatura de um iceberg. Era
impossível, fui forçada a desistir.

Eu esperava descansar um pouco nessa primeira noite em um hospício, mas
estava fadada a me decepcionar. Quando as enfermeiras noturnas chegaram,
elas estavam curiosas em me ver e descobrir como eu era. Logo que
saíram, ouvi alguém na minha porta perguntando por Nellie Brown. Comecei
a tremer, sempre com medo de que minha sanidade seria revelada.
Escutando a conversa, descobri que era um repórter atrás de mim. Ouvi
ele pedir minha trouxa de roupas para que pudesse analisá-la. Continuei
escutando a conversa sobre mim com ansiedade e fiquei aliviada ao
descobrir que era considerada uma louca incurável. Fiquei encorajada.
Depois que o repórter foi embora, ouvi a chegada de novas pacientes e
também que um doutor estava presente e pretendia falar comigo. Não sabia
por que motivo ele queria me ver e fiquei imaginando coisas horríveis,
exames e tudo mais. Quando entraram no meu quarto, eu tremia por mais do
que medo.

--- Nellie Brown, este é o médico. Ele quer falar com você --- a
enfermeira informou. @ Se isso era tudo que ele queria, decidi que era
capaz de suportar o encontro. Removi o cobertor com o qual tapara minha
cabeça de susto e olhei para cima. A visão era encorajadora.

O médico era um homem jovem e bonito, com os ares e o tom de um
cavalheiro. Há quem tenha criticado essa ação, mas tenho certeza que,
mesmo que tenha sido um tanto indiscreto, o jovem doutor tinha apenas
boas intenções para comigo. Ele deu um passo, sentou-se na ponta da cama
e colocou o braço sobre meus ombros. Interpretar o papel de louca para
esse homem era uma tarefa terrível, e apenas uma menina consegue se
solidarizar com a minha situação.

--- Como vai esta noite, Nellie? --- ele perguntou, simpático.

--- Estou bem.

--- Mas você está doente --- ele disse.

--- Estou? --- respondi, e virei minha cabeça no travesseiro para
sorrir.

--- Quando você saiu de Cuba, Nellie?

--- Ah, você conhece a minha casa? --- perguntei.

--- Sim, conheço bem. Não lembra de mim? Eu lembro de você.

--- Lembra? --- e, mentalmente, disse que não o esqueceria. @ Ele estava
acompanhado de um amigo que nunca abriu a boca, apenas me observou
enquanto fiquei deitada. Depois de diversas perguntas, todas as quais
respondi honestamente, ele me deixou. Logo vieram outros problemas, no
entanto. As enfermeiras passaram a noite inteira lendo em voz alta umas
para as outras e as outras pacientes, assim como eu, não conseguiram
dormir. Elas marchavam a cada hora ou meia hora, as botas ressoando como
as de um batalhão de cavalaria pelo corredor, e examinavam cada uma das
pacientes. Obviamente, isso nos ajudava a permanecer acordadas. Quando a
manhã se aproximou, elas começaram a bater ovos para o café da manhã e o
som fez com que percebesse a fome horrível que estava sentindo. De
tempos em tempos, ouvíamos berros e choros da ala masculina, o que não
ajudou a passar a noite mais alegremente. Finalmente, a campainha da
ambulância que trazia mais infelizes soou, dando o sinal para o fim da
vida e da liberdade. Assim passei minha primeira noite como uma das
insanas em Bellevue.

\label{section-4}

\chapter{O objetivo em
vista}\label{capuxedtulo-vii-o-objetivo-em-vista}

Às seis da manhã de domingo, 25 de setembro, as enfermeiras arrancaram
meus cobertores. @--- Anda, está na hora de sair da cama --- elas
disseram, abrindo a janela para deixar entrar um vento gelado. @ Minhas
roupas foram devolvidas. Depois de me vestir, fui levada a uma pia, onde
todas as outras pacientes tentavam arrancar de seus rostos os sinais do
sono. Às sete, recebemos um prato horrível que, segundo Mary nos
informou, seria caldo de galinha. O frio, que já havia nos feito sofrer
o suficiente no dia anterior, era cruel. Quando reclamei para a
enfermeira, esta respondeu que umas das regras da instituição era não
ligar o aquecimento até outubro, então precisaríamos suportá-lo, já que
a tubulação de vapor sequer estava instalada. A seguir, as enfermeiras
da noite se armaram com tesouras e começaram a manicurar as pacientes.
Elas cortaram minhas unhas bem curtinhas, assim como as de várias outras
pacientes. Logo depois disso, um médico jovem e bonito apareceu e eu fui
levada à sala de estar.

--- Qual é o seu nome? --- ele perguntou.

--- Nellie Moreno --- respondi.

--- Então por que deu o sobrenome de Brown? --- ele perguntou. --- O que
há de errado com você?

--- Nada. Eu não queria vir para cá, fui trazida. Quero ir embora. Você
vai me deixar sair?

--- Se sairmos, você vai ficar comigo? Não vai fugir de mim quando
chegar na rua?

--- Não posso prometer que não vou --- respondi, sorrindo e suspirando,
pois ele era bonito.

O doutor fez várias outras perguntas. Eu enxergava rostos na parede?
Ouvia vozes? Respondi como pude.

--- Você ouve vozes à noite? --- ele perguntou.

--- Sim, conversam tanto que não consigo dormir.

--- Achei que sim --- ele disse consigo mesmo, e então me perguntou: ---
O que dizem essas vozes?

--- Bem, eu nem sempre escuto. Às vezes, muitas vezes, elas falam de
Nellie Brown, mas também de outros assuntos que não me interessam tanto
--- respondi honestamente.

--- Isso basta --- ele disse para a Srta. Scott, que estava no lado de
fora.

--- Posso ir embora? --- perguntei.

--- Sim --- ele respondeu, rindo satisfeito. --- Logo vamos mandá-la
embora.

--- Aqui é muito frio, eu quero sair.

--- É verdade --- ele disse para a Srta. Scott. --- O frio aqui dentro é
quase insuportável, você vai ter casos de pneumonia se não tomar
cuidado.

Com isso, fui levada da sala e outra paciente entrou. Sentei ao lado da
porta e esperei para escutar como ele testaria da sanidade das outras
pacientes. Com poucas variações, o exame foi exatamente igual ao meu.
Todas as pacientes ouviram as mesmas perguntas: se viam rostos nas
paredes, se ouviam vozes, o que elas diziam. Também devo acrescentar que
todas as pacientes negaram essas aberrações visuais e auditivas. Às dez
horas, recebemos uma caneca de caldo de carne sem sal; ao meio-dia, um
pedaço de carne fria e uma batata; às três, uma xícara de mingau de
aveia; às cinco e meia, uma xícara de chá e uma fatia de pão sem
manteiga. Estávamos todas com frio e famintas. Depois que o médico foi
embora, as enfermeiras nos deram xales e nos mandaram caminhar de um
lado para o outro nos corredores para nos aquecermos. Durante o dia, o
pavilhão foi visitado por diversas pessoas que tinham curiosidade em ver
a maluca de Cuba. Mantive minha cabeça coberta, alegando estar com frio,
temendo que algum dos repórteres me reconhecesse. Alguns visitantes
estavam em busca de uma menina desaparecida, então várias vezes foi
pedido que eu retirasse o xale. Depois de me olharem, eles diziam ``não
conheço ela'' ou ``não é ela'', pelo qual ficava secretamente grata. O
Diretor O'Rourke me visitou e testou suas habilidades em um exame.
Depois, em diversas ocasiões, ele trouxe algumas mulheres bem-vestidas e
cavalheiros para observar a misteriosa Nellie Brown.

Os repórteres foram o maior incômodo. Eram tantos! E eram todos tão
espertos, fiquei morrendo de medo que algum deles percebesse minha
sanidade. Eles foram bondosos e simpáticos comigo, e sempre muito gentis
nas suas perguntas. Minha visita da noite interior veio até a janela
enquanto alguns repórteres me entrevistavam na sala de estar e disse à
enfermeira que permitisse que se encontrassem comigo, pois eles
ajudariam a encontrar alguma pista sobre a minha identidade.

Durante a tarde, o dr. Field veio me examinar. Ele fez poucas perguntas,
uma das quais não tinha nenhuma relevância para um caso como o meu. A
pergunta principal foi sobre meu lar e meus amigos, se eu tivera
namorados ou já fora casada. A seguir, ele me mandou esticar os braços e
mover os dedos, o que fiz sem a menor hesitação, mas depois ouvi ele
dizer que meu caso não tinha solução. As outras pacientes ouviram as
mesmas perguntas.

Quando o médico estava prestes a deixar o pavilhão, a Srta. Tillie
Mayard descobriu que estava na ala dos insanos. Ela foi até o dr. Field
e perguntou por que fora enviada para lá.

--- Você acaba de descobrir que está em um hospício? --- o doutor
perguntou.

--- Sim. Meus amigos disseram que estavam me mandando para uma ala de
convalescentes, onde seria tratada para a debilidade nervosa que tenho
sofrido desde minha doença. Quero sair deste lugar imediatamente.

--- Bem, você não vai sair tão cedo --- ele disse com uma risadinha.

--- Se você sabe de alguma coisa, então vai perceber que eu sou
perfeitamente sã --- ela respondeu. --- Por que não aplica um teste?

--- Já sabemos tudo que precisamos nessa questão. @ Com isso, o médico
foi embora, deixando aquela pobre mulher condenada ao hospício,
provavelmente pelo resto da vida, sem lhe dar a menor chance de provar
sua sanidade.

A noite de domingo apenas repetiu a de sábado. Passei a madrugada
inteira acordada por causa da conversa das enfermeiras e de seus passos
pesados pelos corredores sem tapetes. Na manhã de segunda-feira, fomos
informadas que seríamos levadas à uma e meia. As enfermeiras me
perguntaram várias e várias vezes onde morava e todas pareciam ter a
ideia de que um amante havia me expulsado e arruinado meu cérebro.
Vários repórteres apareceram naquela manhã, trabalhando sem descansar
para obter alguma novidade. A Srta. Scott se recusou a deixar que me
vissem, no entanto, ao que fiquei grata. Se tivessem tido livre acesso à
minha pessoa, meu mistério provavelmente não teria durado muito tempo,
pois vários deles me conheciam de vista. O Diretor O'Rourke apareceu
para uma última visita e uma breve conversa comigo. Ele escreveu seu
nome em meu caderno, dizendo à enfermeira que eu me esqueceria de tudo
em uma hora. Sorri e pensei que não tinha certeza disso. Outras pessoas
vieram me visitar, mas nenhuma delas me conhecia ou sabia dar alguma
informação sobre mim.

Era meio-dia. Quanto mais se aproximava o momento de partir para a ilha,
mais nervosa ficava. Eu temia cada nova chegada, com medo de ter meu
segredo revelado no último instante. Entregaram-me um xale e meu chapéu
e luvas. Mal consegui vesti-los, tal era o estado dos meus nervos.
Finalmente chegou o atendente e eu dei adeus a Mary, colocando ``umas
moedinhas'' em suas mãos. @--- Deus lhe abençoe --- ela disse. --- Vou
rezar por você. Alegre-se, moça. Você é jovem, vai superar isso. @ Eu
respondi que esperava que sim e então dei adeus à Srta. Scott em
espanhol. O atendente grosseiro torceu seus braços em torno dos meus e
me levou, quase arrastou, até a ambulância. Um grupo de estudantes se
reunira para nos admirar. Coloquei o xale sobre meu rosto e me acomodei
no coche, agradecida. A Srta. Neville, Srta. Mayard, sra. Fox e sra.
Schanz foram colocadas uma a uma depois de mim. Um homem entrou conosco,
as portas foram trancadas e o coche atravessou o portão em grande
estilo, até o Hospício e a vitória! As pacientes não cometeram nenhuma
tentativa de fuga. O hálito do atendente bastava para deixar qualquer um
tonto.

Quando chegamos ao cais, tamanha era a multidão em torno do coche que a
polícia precisou ser chamada para afastá-la e nos deixar alcançar o
barco. Fui a última da procissão. Fui escoltada pela prancha e a brisa
fresca do rio soprou o bafo de uísque do atendente até me fazer
cambalear. Fui levada até uma cabine suja, onde minhas companheiras
estavam sentadas em um banco estreito. As janelinhas estavam fechadas e,
com o cheiro daquele quarto imundo, a atmosfera era sufocante. Em uma
das pontas da cabine, um pequeno beliche estava em condição tão terrível
que precisei segurar meu nariz quando me aproximei. Uma menina doente
foi colocada nele. Uma senhora idosa, com um toucado enorme e uma cesta
suja cheia de nacos de pão e restos de carne, completava nossa
companhia. A porta era guardada por duas atendentes. A primeira usava um
vestido feito de forro de colchão, a segunda com alguma tentativa de
estilo. Eram mulheres gigantescas e vulgares, escarrando suco de tabaco
no convés com mais habilidade do que charme. Uma dessas criaturas
terríveis parecia ter muita fé no poder do olhar dirigido contra os
insanos, pois quando qualquer uma de nós se mexia ou se levantava para
olhar pela janela, a atendente dizia ``sentada'', franzia o cenho e nos
encarava com um olhar simplesmente aterrorador. Enquanto guardavam a
porta, elas conversavam com alguns homens do lado de fora, discutindo o
número de pacientes e suas próprias vidas de um modo que não tinha nada
de salutar ou de refinado.

\imagemmedia{}{./media/rId43.png}O
barco parou e a velha senhora e a menina doente foram retiradas. O resto
de nós recebeu a ordem de ficarmos sentadas. Na próxima parada, minhas
companheiras foram tiradas do barco, uma a uma. Eu fui a última, e
pareceu necessário que um homem e uma mulher me conduzissem pela prancha
até a praia. Uma ambulância estava a nossa espera, e dentro dela as
quatro outras pacientes.

--- O que é esse lugar? --- perguntei ao homem, que tinha os dedos
cravados no meu braço.

--- A Ilha de Blackwell, um lugar louco de onde você nunca vai sair.

Com isso, fui atirada para dentro da ambulância, o trampolim foi
colocado, um oficial e um carteiro saltaram na traseira e eu fui levada
rapidamente em direção ao Hospício da Ilha de Blackwell.

\chapter{Dentro do
manicômio}\label{capuxedtulo-viii-dentro-do-manicuxf4mio}

Enquanto o coche atravessava rapidamente o lindo gramado que levava ao
hospício, meu sentimento de satisfação por ter atingido meu objetivo foi
sendo refreado pelo olhar deprimido nos rostos de minhas companheiras.
As pobres mulheres não tinham esperança alguma de uma salvação imediata.
Sem culpa nenhuma, elas estavam sendo levadas à prisão, provavelmente
perpétua. Em comparação, seria muito mais fácil subir à forca do que
entrar nessa tumba de horrores vivos! O coche foi adiante e eu, assim
como minhas camaradas, dei um último adeus desesperado à liberdade
quando avistamos os longos edifícios de pedra. Quando passamos por um
dos prédios baixos, o fedor foi tão horrível que fui forçada a prender a
respiração; mentalmente, decidi que era a cozinha. Mais tarde, descobri
que estava correta em minha dedução, e sorri quando enxerguei a placa no
final da trilha: ``Proibidos visitantes nesta estrada''. Creio que a
placa seria desnecessária se os visitantes tentassem seguir a estrada
uma vez que fosse, especialmente em um dia de calor.

O coche parou e a enfermeira e o oficial encarregado nos mandaram
descer. @--- Graças a Deus, vieram quietas! --- a enfermeira completou.
@ Obedecemos à ordem de subir um lance de degraus de pedra estreitos,
evidentemente destinados àquelas pessoas que gostam de subir escadas de
três em três degraus. Curiosa em saber se minhas companheiras sabiam
onde estávamos, perguntei à Srta. Tillie Mayard:

--- Onde estamos?

--- No Hospício de Lunáticos da Ilha de Blackwell --- ela respondeu
tristonha.

--- Você é louca? --- perguntei.

--- Não, mas como fomos mandadas para cá, precisamos ficar quietas até
descobrirmos algum jeito de fugir. Não vai ser fácil, no entanto, se
todos os médicos forem como o dr. Field e se recusarem a me ouvir e não
me derem nenhuma chance de provar minha sanidade. @ Fomos levadas até um
vestíbulo estreito e a porta foi trancada atrás de nós.

\imagemmedia{}{./media/rId45.png}Apesar
da certeza sobre minha própria sanidade e da garantia de que seria solta
em alguns dias, meu coração sentiu uma pontada. Declarada insana por
quatro médicos especialistas e trancada atrás das grades implacáveis de
um manicômio! Não ser confinada sozinha, mas ter por companhia, noite e
dia, lunáticas insensatas e tagarelas; dormir com elas, comer com elas,
ser considerada uma delas era uma posição nada confortável. Com muita
timidez, seguimos as enfermeiras pelo corredor sem tapetes até uma sala
cheia de mulheres supostamente loucas. Ela nos mandou sentar e algumas
pacientes fizeram a bondade de abrir espaço para nós. Elas nos olharam
com curiosidade e uma se aproximou de mim.

--- Quem mandou você para cá? --- ela perguntou.

--- Os doutores --- respondi.

--- Por quê? --- ela insistiu.

--- Bem, eles dizem que sou insana --- admiti.

--- Insana! --- ela repetiu, incrédula. --- Você não tem cara de louca.

Essa mulher era esperta demais, concluí, e fiquei contente em responder
as ordens berradas de seguir a enfermeira e ver o doutor. Essa
enfermeira, a propósito, a Srta. Grupe, tinha um rosto alemão simpático,
e se eu não tivesse detectado certos traços de dureza em sua boca, teria
esperado, como fizeram minhas companheiras, receber apenas bondade dela.
A enfermeira nos deixou sozinhas em uma pequena sala de espera no fim do
corredor enquanto entrou no escritório que se abria para a recepção.

--- Gosto de ir até o coche --- ela disse para o interlocutor invisível
no lado de dentro. --- Ajuda a quebrar a rotina. @ Ele respondeu que o
ar livre ajudava sua aparência e ela voltou para nós com um sorriso
afetado.

\imagemmedia{}{./media/rId46.png}---
Tillie Mayard, venha cá --- ela disse. @ A Srta. Mayard obedeceu. Apesar
de não ver o interior do escritório, eu conseguia ouvi-la defender seu
caso em um tom gentil, mas firme. Todas as suas afirmações eram
absolutamente racionais e imaginei que médico algum deixaria de se
impressionar com sua história. Ela contou sobre a doença recente e que
sofria de uma debilidade nervosa. Ela implorou que realizassem todos os
testes de insanidade que tinham, se tinham algum, e que lhe dessem
justiça. Pobrezinha, como eu tinha pena dela! Decidi naquele instante
que minha missão no futuro seria usar todos os meios ao meu dispor para
ajudar minhas irmãs em sofrimento, que mostraria como elas são
trancafiadas sem um julgamento justo. Sem ouvir uma palavra de
solidariedade ou encorajamento, ela foi trazida de volta aonde estávamos
sentadas.

A sra. Louise Schanz foi levada até o dr. Kinier.

--- Seu nome? --- ele perguntou em voz alta. @ Ela respondeu em alemão,
dizendo que não falava inglês e que não o entendia. Contudo, quando ele
disse ``Sra. Louise Schanz'', ela respondeu ``ja, ja''. A seguir, ele
tentou outras perguntas, mas quando viu que ela não entendia uma palavra
que fosse de inglês, se virou para a Srta. Grupe e disse:

--- Você é alemã, fale com ela por mim.

A Srta. Grupe se revelou uma daquelas pessoas que têm vergonha da
própria nacionalidade e se recusou, dizendo que entendia muito pouco do
idioma materno.

--- Eu sei que você fala alemão. Pergunte a essa mulher o que o marido
faz da vida --- e ambos riram, como se ele tivesse feito uma piada.

--- Eu falo muito pouco --- ela reclamou, mas finalmente conseguiu
descobrir a profissão do sr. Schanz.

--- E de que adiantou mentir para mim? --- o doutor continuou, com um
sorriso que negava a grosseria da pergunta.

--- Não posso falar mais nada --- ela disse, e não falou.

Foi dessa maneira que a sra. Louise Schanz acabou confinada ao hospício,
sem a chance de se fazer compreendida. Qual seria a desculpa para
tamanho descuido quando é tão fácil obter um intérprete? Se o
confinamento fosse por apenas alguns dias, seria possível questionar sua
necessidade. Mas aquela mulher fora levada do mundo livre para um
hospício sem dar seu consentimento e lá não teve nenhuma chance de
provar sua sanidade. Muito provavelmente presa pelo resto da vida atrás
das grades do hospício, sem mesmo ser informada do porquê em seu próprio
idioma. Compare o caso com o de um criminoso, que recebe todas as
chances de provar sua inocência. Quem não preferiria ser um assassino e
correr o risco do que ser declarado insano, sem nenhuma esperança de
fugir? A sra. Schanz implorou em alemão para saber onde estava, implorou
por liberdade. Entre soluços, ela foi levada de volta até nós sem jamais
ser ouvida.

\imagemmedia{}{./media/rId47.png}A
seguir, foi a vez da sra. Fox de passar por esse exame fraco e trivial e
sair do escritório condenada, então a Srta. Annie Neville. Fui deixada
em último lugar mais uma vez. A essa altura, estava decidida a agir da
mesma maneira que quando em liberdade, exceto que me recusaria a contar
quem era ou onde morava.

\label{section-5}

\chapter{Um especialista(?)
trabalhando}\label{capuxedtulo-ix-um-especialista-trabalhando}

--- Nellie Brown, o doutor quer falar com você --- a Srta. Grupe disse.
@ Eu entrei e me mandaram sentar em frente ao dr. Kinier, junto à
escrivaninha.

--- Qual é o seu nome? --- ele perguntou sem erguer os olhos.

--- Nellie Brown --- respondi de imediato.

--- Onde você mora? --- ele continuou, escrevendo o que eu dissera em um
livro enorme.

--- Em Cuba.

--- Ah! @ Entendendo subitamente o que estava acontecendo, ele se virou
para a enfermeira:

--- Viu alguma coisa sobre ela nos jornais?

--- Sim --- ela respondeu. --- Vi uma reportagem sobre essa menina no
\emph{Sun} de domingo.

--- Fique com ela aqui enquanto eu vou no escritório ver a notícia de
novo.

Ele nos deixou e fui roubada de meu chapéu e xale. Quando voltou, ele
disse que não conseguira achar o jornal, mas recontou para a enfermeira
o que lembrava a história de minha estreia.

--- Qual é a cor dos olhos dela?

A Srta. Grupe olhou e respondeu ``cinza'', apesar de sempre terem me
dito que meus olhos eram castanhos ou castanhos esverdeados.

--- Quantos anos você tem? --- ele perguntou. @--- Fiz dezenove em maio.
@--- Quando é o seu próximo passe? @ A pergunta fora feita para a
enfermeira; pelo que entendi, o termo se referia a sua próxima folga.

--- Sábado que vem --- ela disse, rindo.

--- Você vai até a cidade? @ Ambos riram quando ela respondeu que sim.

@--- Meça ela --- ele mandou. @ Fui colocada sob um medidor de estatura,
que foi apertado contra minha cabeça.

--- Qual é o resultado? --- o doutor perguntou.

--- Você sabe que eu não sei --- ela disse.

--- Sabe sim, vá em frente. Qual é a altura?

--- Não sei. Tem uns números ali, mas não sei o que são.

--- Sabe sim. Olhe de novo e me diga.

--- Não consigo, olhe você mesmo. @ E os dois riram de novo. O médico
saiu de trás da escrivaninha e se aproximou para conferir o resultado.

--- Um metro e sessenta e cinco centímetros, está vendo? --- ele disse,
tomando sua mão e indicando os números.

Pela voz, eu soube que ela ainda não entendia, mas nada disso me
preocupava, pois o médico parecia ter prazer em ajudá-la. Em seguida,
fui colocada sobre a balança e a enfermeira remexeu o aparelho até os
pesos se equilibrarem.

--- Quanto? --- o médico perguntou, voltando para a escrivaninha.

--- Não sei. Vai ter que ver por si mesmo --- ela respondeu, chamando-o
pelo primeiro nome, que esqueci. @ Ele se virou e, também chamando-a
pelo prenome, disse:

--- Você está ficando muito saidinha! --- e os dois riram. @ Disse o
peso à enfermeira (51 kg), que por sua vez informou o doutor.

--- Que horas você vai jantar? --- ele perguntou e ela respondeu. @ O
médico deu mais atenção à enfermeira do que a mim e fez seis perguntas a
ela para cada uma que fez a mim. Por fim, ele anotou meu destino no
livro à sua frente. @--- Não estou doente e não quero ficar aqui ---
declarei. --- Ninguém tem o direito de me prender desta maneira. @ Ele
não deu atenção à minha fala. Tendo completado suas anotações, e também
sua conversa com a enfermeira naquela sessão, disse que aquilo bastava.
Junto com minhas companheiras, fui levada de volta à sala de estar.

--- Você toca piano? --- elas perguntaram.

--- Sim, sim, desde criancinha --- respondi.

Elas insistiram que eu tocasse, posicionando-me em um banco de madeira
perante um piano quadrado antiquado. Toquei algumas notas e a resposta
desafinada foi um rangido que ecoou nos meus ossos.

--- Que horrível --- exclamei, e virei-me para a enfermeira ao meu lado,
a Srta. McCarten. --- Nunca vi um piano tão desafinado.

--- Mas que peninha --- ela disse, maldosa. --- Vamos ter que encomendar
um especialmente para você.

Comecei a tocar as variações de ``Home! Sweet Home!'' A conversa parou e
todas as pacientes ficaram sentadas em silêncio enquanto meus dedos
gelados se moviam lenta e dolorosamente sobre o teclado. Terminei a
canção, desajeitada, e recusei todos os pedidos de tocar mais. Sem ter
um lugar disponível onde sentar, continuei ocupando a cadeira em frente
ao piano enquanto avaliava o ambiente.

Era uma sala longa e vazia, com bancos amarelos simples ao redor. Esses
bancos, todos perfeitamente retos, e igualmente desconfortáveis, tinham
espaço para seis pessoas, apesar de em quase todos os casos seis estarem
se amontando neles. As janelas gradeadas, erguidas a cerca de um metro e
meio do chão, davam para as duas portas duplas que levavam ao corredor.
A brancura das paredes era aliviada em parte por três litografias, uma
de Fritz Emmet e as outras de músicos negros. No centro da sala havia
uma grande mesa coberta com um cobertor branco, em torno da qual ficavam
sentadas as enfermeiras. Tudo estava limpíssimo e pensei que as
enfermeiras deviam ser excelentes trabalhadoras, para manter o lugar tão
em ordem. Alguns dias depois, ri da minha própria estupidez, pensando
que as enfermeiras faziam algum trabalho. Quando estas descobriram que
eu não iria mais tocar, a Srta. McCarten veio até mim.

--- Saia já daqui --- ela exclamou com grosseria, e então fechou o
piano.

--- Brown, venha cá --- foi a próxima ordem que recebi, de uma mulher
vulgar e de rosto avermelhado sentada junto à mesa. --- O que você está
vestindo?

--- Minhas roupas --- respondi.

Ela ergueu meu vestido e minhas saias e anotou: um par de sapatos, um
par de meias, um tecido de pano, um chapéu de marinheiro de palha e
assim por diante.

\chapter{Meu primeiro
jantar}\label{capuxedtulo-x-meu-primeiro-jantar}

Depois que esse exame terminou, ouvimos alguém gritar ``vão para o
corredor!'' Uma das pacientes teve a bondade de nos explicar que este
era o convite para o jantar. Nós, as atrasadas, tentamos acompanhar,
então entramos no corredor e ficamos paradas junto à porta onde todas as
mulheres estavam aglomeradas. Como tremíamos! As janelas estavam abertas
e uma corrente de vento atravessava o corredor. As pacientes estavam
azuis de frio e os minutos foram se acumulando até somarem um quarto de
hora. Por fim, uma das enfermeiras apareceu para destrancar a porta, que
nos levou até a plataforma no final da escadaria. Mais uma vez, ficamos
paradas diretamente em frente a uma janela aberta.

--- Que imprudência das atendentes, deixar todas essas mulheres
mal-vestidas aqui nesse frio --- a Srta. Neville disse.

--- É de uma brutalidade horrível --- completei enfaticamente, olhando
para as pobres loucas prisioneiras que tremiam de frio. @ Observando
aquelas mulheres, pensei que não teria um jantar agradável. Elas
pareciam tão perdidas, tão desesperançadas. Algumas tagarelavam coisas
sem sentido para interlocutores invisíveis, outras riam ou choravam sem
motivo. Uma velha senhora grisalha me cutucava e, com piscadelas e
acenos sábios com a cabeça e gestos patéticos com os olhos e as mãos, me
garantia que não deveria dar bola para aquelas pobres criaturas. Eram
todas loucas, afinal. @--- Parem junto ao aquecedor --- veio a ordem.
--- E façam fila de duas em duas. @--- Mary, arranje uma companheira.
@--- Quantas vezes vou ter que mandar ficarem na fila? @--- Parada. @
Enquanto as ordens eram anunciadas, empurrões e puxões eram
administrados entre as pacientes, muitas vezes com um safanão na orelha.
Depois dessa terceira e última parada, fomos levadas até uma sala de
jantar longa e estreita, onde todas correram para a mesa.

A mesa ia de uma ponta à outra da sala e era descoberta e
desconvidativa. As pacientes se sentavam em bancos longos sem costas
sobre os quais precisavam se arrastar para ficarem de frente para a
mesa. Lado a lado sobre a mesa havia diversas tigelas enormes recheadas
de alguma coisa rosada que as pacientes diziam ser chá. Ao lado de cada
tigela estava um pedaço de pão, cortado em fatias grossas cobertas de
manteiga. Um pequeno pires com cinco uvas passas acompanhava o pão. Uma
mulher gorda se adiantou, arrancou vários pires das pacientes ao seu
redor e despejou as passas no próprio. Depois, segurando firme sua
própria tigela, ergueu outra e engoliu seu conteúdo com um gole só. Em
seguida, ela fez o mesmo com uma segunda tigela em menos tempo do que
seria necessário para contar a história. Fiquei tão admirada com aquele
roubo que quando olhei para minha própria parcela daquela refeição, a
mulher à minha frente, sem ao menos pedir licença, agarrou meu pão e me
deixou sem nada.

Quando viu isso acontecer, outra paciente me ofereceu o seu, mas
agradeci a bondade e fui pedir mais para a enfermeira. Ela atirou um
pedaço sobre a mesa e fez uma piada, dizendo que eu esquecera onde
morava, não como se come. Experimentei o pão, mas a manteiga era
intragável. Uma menina alemã no outro lado da mesa me informou que eu
poderia comer o pão sem nada se quisesse e que poucas pacientes
conseguiam comer a manteiga. Voltei minha atenção para as uvas passas e
descobri que algumas poucas seriam mais do que suficientes. Uma paciente
pediu que eu lhe desse as minhas, e foi o que fiz. Meu chá era tudo que
sobrara. Tomei um gole, e um gole bastou. O chá não tinha açúcar e
parecia ter sido feito em cobre. De tão fraco, era quase água. Entreguei
a tigela para outra paciente faminta, apesar da advertência que a Srta.
Neville me deu:

--- Você precisa comer à força ou vai acabar doente --- ela disse. --- E
num lugar desses, você é capaz até de ficar louca. Para cuidar do
cérebro é preciso cuidar do estômago.

--- É impossível comer aquilo --- respondi e, apesar de sua insistência,
não comi nada naquela noite.

As pacientes não precisaram de muito tempo para consumir tudo que havia
de comestível sobre a mesa e logo recebemos a ordem de nos enfileirarmos
no corredor. Quando terminamos de nos organizar, as portas foram
destrancadas e a nova ordem foi de nos dirigirmos para a sala de estar.
Muitas das pacientes ficaram ao nosso redor e mais uma vez tanto elas
quanto as enfermeiras insistiram que eu tocasse piano. Para agradar as
pacientes, prometi tocar, enquanto a Srta. Tillie Mayard cantaria. A
primeira canção que ela me pediu foi ``Rock-a-bye Baby'', o que atendi.
Seu canto foi maravilhoso.

\chapter{No banho}\label{capuxedtulo-xi-no-banho}

Mais algumas canções e nos mandaram acompanhar a Srta. Grupe. Fomos
levadas a um banheiro frio e úmido e recebemos a ordem de nos despir. Eu
reclamei? Nunca me esforcei tanto na vida quanto na tentativa de
recusar. Disseram que se não me despisse, elas usariam a força e que o
resultado não seria nada gentil. Foi quando notei uma das mulheres mais
loucas da ala parada junto à banheira cheia, segurando um trapo
desbotado nas mãos. Ela matraqueava consigo mesma e soltava gargalhadas
quase demoníacas. Agora eu sabia qual seria meu destino. Fiquei
arrepiada. Começaram a me despir, tirando minhas roupas, uma a uma.
Finalmente, me sobrou apenas uma peça. @--- Não vou removê-la --- disse
com veemência, mas ela a arrancaram. @ Olhei para o grupo de pacientes
que estava reunido junto à porta, assistindo a cena, e saltei para
dentro da banheira com mais energia do que delicadeza.

\imagemmedia{}{./media/rId52.png}A
água estava gelada e eu comecei a reclamar. Como foi inútil! Implorei
que pelo menos as outras pacientes fossem levadas embora, mas a resposta
foi me mandarem calar a boca. A louca começou a me esfregar. Não consigo
pensar em uma palavra melhor para que ela fez, apenas esfregar. Ela
tirou um pouco de sabão líquido de uma panelinha de lata e passou-o em
mim, até mesmo meu rosto e meu cabelo. Eu não conseguia mais ver nem
falar nada, apesar de ter implorado que meu cabelo fosse deixado em paz.
A velha continuou com o esfrega-esfrega, sempre tagarelando sozinha.
Meus dentes batiam e meus membros estavam arrepiados e azuis de frio. De
repente, um depois do outro, três baldes de água gelada foram despejados
sobre minha cabeça, nos meus olhos, minhas orelhas, meu nariz, minha
boca. Creio que senti algumas das sensações de quem se afoga quando me
arrastaram, tremendo e arfando, de dentro da banheira. Pela primeira
vez, eu parecia realmente louca. Avistei o olhar indescritível nos
rostos de minhas companheiras, que haviam testemunhado meu destino e
sabiam que logo sofreriam o mesmo. Incapaz de me controlar pensando na
figura absurda que apresentava, comecei a gargalhar. Ainda ensopada, me
colocaram em uma camisa fina e curta de flanela em cuja extremidade
estava escrito, em letras negras enormes, ``Hospício, I. B., S. 6'',
significando Ilha de Blackwell, Salão 6.

A essa altura, a Srta. Mayard estava despida. Por mais que tivesse
odiado meu banho, teria tomado outro se pudesse tê-la poupado da mesma
experiência. Imagine mergulhar uma menina doente em um banho gelado,
quando eu, que nunca ficara doente na vida, fiquei tremendo como se
estivesse febril. Ouvi ela explicar para a Srta. Grupe que sua cabeça
ainda doía por causa da doença. Seu cabelo era curto, e boa parte tinha
caído, então ela pediu que a louca a esfregasse com mais delicadeza.

--- Você não vai se machucar --- a Srta. Grupe retrucou. --- Agora cale
a boca ou vai ser pior. @ A Srta. Mayard se calou, e foi a última vez
que a vi naquela noite.

Fui levada a um quarto onde havia seis camas e fui deitada em uma delas
quando alguém apareceu e me arrancou dali.

--- Nellie Brown tem que ficar sozinha no quarto hoje à noite, parece
que ela é barulhenta.

Fui levada ao quarto 28, com a missão de me acomodar na cama. Era uma
tarefa impossível. A cama fora arrumada alta no centro, com ambos os
lados caídos. Ao primeiro toque, minha cabeça deixou o travesseiro
inundado e minha roupa molhada transferiu parte da umidade para o
lençol. Quando a Srta. Grupe entrou, perguntei se não poderia receber
uma camisola.

--- Não temos esse tipo de coisa nesta instituição --- ela disse.

--- Não gosto de dormir sem --- respondi.

--- Bem, eu não me importo --- ela disse. --- Você está em uma
instituição pública, não pode esperar por nada. Isto aqui é caridade,
você deveria ficar grata pelo que receber.

--- Mas a cidade paga pela manutenção desses lugares --- insisti. --- E
ela paga pessoas para serem boas com as infelizes trazidas para cá.

--- Bem, melhor não esperar bondade por aqui, porque não vai encontrar
--- ela disse, e então saiu e fechou a porta.

\imagemmedia{}{./media/rId53.png}Eu
tinha um lençol e um oleado por baixo e um lençol e um cobertor de lã
negra por cima. Nunca senti nada tão incômodo quanto aquele cobertor de
lã enquanto tentava mantê-lo ao redor de meus ombros para afastar o
frio. Quando o puxava para cima, meus pés ficavam expostos, mas quando
puxava para baixo, era a vez dos meus ombros. Não havia absolutamente
nada no quarto além da cama e de mim. Como a porta fora trancada,
imaginei que passaria a noite sozinha, mas ouvi os passos pesados de
duas mulheres no corredor. Elas paravam em frente às portas, as
destrancavam e, depois de alguns instantes, eu as ouvia retrancá-las.
Tudo isso aconteceu sem a menor tentativa de fazer silêncio, em todas as
portas do outro lado do corredor até o meu quarto. Quando chegaram em
mim, elas pararam. A chave foi inserida na fechadura e virada. Olhei
para as duas prestes a entrar. Elas usavam vestidos listrados de branco
e marrom com botões de bronze, grandes aventais brancos, um cordão verde
pesado em torno da cintura no qual um molho de chaves enormes ficava
pendurado e boininhas brancas sobre suas cabeças. Vestidas da mesma fora
que as atendentes diurnas, eu sabia que eram as enfermeiras. A primeira
levava uma lanterna.

--- Esta é Nellie Brown --- ela disse para a assistente, virando a luz
para o meu rosto.

--- Quem é você? --- perguntei, olhando para ela.

--- Eu sou a enfermeira noturna, queridinha --- ela respondeu e,
desejando que eu dormisse bem, saiu e trancou a porta. @ Elas entraram
no meu quarto várias vezes durante a noite. Mesmo que eu tivesse
conseguido dormir, o ato de destrancar a porta pesada, sua conversa alta
e os passos pesados teriam me acordado.

Não conseguia dormir, então fiquei deitada na cama, imaginando comigo
mesma os horrores que aconteceriam caso houvesse um incêndio no
hospício. Todas as portas tinham trancas separadas e as janelas tinham
grades pesadas, o que tornava a fuga impossível. Segundo o dr. Ingram,
em cada edifício havia cerca de trezentas mulheres. Elas ficavam
confinadas, de uma a dez em cada quarto. É impossível sair a menos que
essas portas sejam destrancadas. Um incêndio não é improvável; pelo
contrário, é uma das ocorrências mais prováveis. Se o edifício pegasse
fogo, as carcereiras ou enfermeiras jamais pensariam em soltar suas
pacientes loucas. Provarei isso posteriormente, quando contar sobre o
tratamento cruel que distribuíam às pobrezinhas sob seus cuidados. Como
estava dizendo, em caso de incêndio, menos de uma dúzia de mulheres
conseguiria escapar. Elas seriam todas deixadas para morrer queimadas.
Mesmo que as enfermeiras fossem bondosas, e não são, seria preciso muito
mais presença de espírito do que as mulheres de sua classe possuem para
enfrentar as chamas e colocar a própria vida em risco para abrir uma
centena de portas e salvar prisioneiras insanas. A menos que uma mudança
seja efetuada, um dia o hospício vai produzir uma história de terror
para superar todas as outras.

Em relação a esse tema, reconto um episódio engraçado que ocorreu logo
antes de minha soltura. Estava conversando com o dr. Ingram sobre
diversos assuntos e finalmente contei o que eu achava que aconteceria em
caso de incêndio.

--- As enfermeiras têm a responsabilidade de abrir as portas --- ele
disse.

--- Mas você com certeza deve saber que elas não esperariam para fazer
isso --- respondi. --- Essas mulheres morreriam queimadas.

Ele ficou em silêncio, incapaz de contradizer minha afirmação.

--- Por que você não faz alguma coisa? --- perguntei.

--- Mas o que eu posso fazer? Faço sugestões até cansar o cérebro, mas
de que adianta? O que você quer que eu faça? --- ele se virou para mim,
a insana declarada.

--- Ora, eu insistiria na instalação de trancas, como as que já vi em
alguns lugares, nas quais o acionamento de uma manivela no final do
corredor permite que todas as portas de um lado sejam travadas ou
destravadas. Com isso, haveria alguma chance de salvação. Hoje, como
toda as portas são trancadas separadamente, a chance é absolutamente
zero.

O dr. Ingram se virou para mim com um olhar ansioso em seu rosto gentil
e perguntou lentamente:

--- Nellie Brown, em quais instituições você fora internada antes de vir
para cá?

--- Nenhuma. Nunca fiquei confinada em nenhuma instituição na vida,
exceto pelo internato.

--- Onde você viu as trancas que descreveu?

Eu as vira na Penitenciária Ocidental em Pittsburgh, Pensilvânia, mas
não ousei informá-lo. Respondi simplesmente:

--- Ah, eu vi em um lugar que estive... quero dizer, que visitei.

--- Só sei de um local que possui essas trancas --- ele disse com
tristeza. --- Sing Sing.

A inferência é clara. Ri alto da acusação implícita e tentei garantir ao
médico que nunca, até então, eu fora prisioneira em Sing Sing, ou sequer
havia visitado a penitenciária.

Quando o sol começou a nascer, finalmente peguei no sono, mas poucos
momentos depois fui acordada com uma grosseria e me mandaram levantar. A
janela foi aberta e minhas roupas foram arrancadas. Meu cabelo ainda
estava molhado e meu corpo todo doía como se eu sofresse de reumatismo.
Algumas peças de roupa foram atiradas no chão e eu recebi a ordem de
vesti-las. Pedi as minhas próprias, mas a Srta. Grady, aparentemente a
enfermeira-chefe, me mandou aceitar o que ganhasse e ficar quieta. Olhei
para as roupas. Uma saia de baixo de algodão escuro grosseiro e um
vestido de chita branco com uma mancha preta. Amarrei os cordões da saia
na cintura e coloquei o vestidinho. Ele era feito, assim como todos
aqueles usados pelas pacientes, com um corpete reto apertado costurado a
uma saia reta. Enquanto abotoava o corpete, percebi que a saia de baixo
era 15 centímetros mais comprida do que a de cima. Por um instante,
precisei me sentar na cama e rir de minha própria aparência. Mulher
alguma jamais desejou um espelho mais do que eu naquele momento.

Vi as outras pacientes atravessando o corredor às pressas, então decidi
não perder nada que pudesse estar acontecendo. Éramos quarenta e cinco
pacientes no Salão 6 e fomos mandadas ao banheiro, onde havia duas
toalhas grosseiras. Vi pacientes loucas que tinham erupções
perigosíssimas por todo o rosto se secando com as toalhas e então vi
mulheres de pele limpa usarem-nas em seguida. Fui à banheira e lavei meu
rosto na torneira, usando a saia de baixo como toalha.

Antes que eu completasse minha ablução, um banco foi trazido para o
banheiro. A Srta. Grupe e a Srta. McCarten entraram com pentes nas mãos.
Elas nos mandaram nos sentar no banco e o cabelo de quarenta e cinco
mulheres foi penteado com uma paciente, duas enfermeiras e seis pentes.
Quando vi algumas das cabeças feridas que estavam sendo penteadas,
pensei que essa era outra dose que não estava no contrato. A Srta.
Tillie Mayard tinha seu próprio pente, mas este foi levado pela Srta.
Grady. Ah, aqueles pentes! Eu nunca entendi a velha expressão ``vou lhe
pentear'', mas aprendi. Meu cabelo, todo emaranhado e úmido por causa da
noite anterior, foi puxado e repuxado. Depois de me queixar inutilmente,
cerrei os dentes e aguentei a dor. Elas se recusaram a me fornecer
grampos, então meu cabelo foi trançado e amarrado com um pedaço de
algodão vermelho esfarrapado. Minha franja cacheada se recusou a
acompanhar o resto, então esse resquício de minha antiga glória, pelo
menos, sobreviveu.

Depois disso, fomos para a sala de estar e tive a chance de procurar
minhas companheiras. Primeiro fiquei perdida, incapaz de diferenciá-las
das outras pacientes, mas depois de algum tempo reconheci a Srta. Mayard
pelo cabelo curto.

--- Como foi seu sono depois do banho frio?

--- Quase congelei, e depois o barulho me deixou acordada. Foi horrível!
Meus nervos já estavam mal antes de eu vir para cá, estou com medo de
não aguentar o cansaço.

Fiz o que pude para animá-la. Pedi que nos dessem mais roupa, pelo menos
o mínimo que o costume determina que as mulheres vistam, mas me mandaram
calar a boca e disseram que já tínhamos tudo que pretendiam nos dar.

Fomos forçadas a acordar às cinco e meia, e às sete e quinze recebemos a
ordem de nos reunir no salão, onde a experiência de espera, assim como
na noite anterior, se repetiu. Quando chegamos à sala de jantar,
finalmente cada uma de nós recebeu uma tigela de chá gelado, uma fatia
de pão com manteiga e um pires com aveia e melaço. Eu estava com fome,
mas a comida era intragável. Pedi uma fatia sem manteiga e recebi. Não
consigo lembrar de nada que tenha a mesma cor negra de sujeira. O pão
era duro e em algumas partes não passava de massa seca. Encontrei uma
aranha na minha fatia, então não comi. Experimentei a aveia com melaço,
mas era horrível, então tentei, sem muito sucesso, engolir o chá.

Depois de voltarmos à sala de estar, várias mulheres receberam a ordem
de arrumar as camas, algumas foram colocadas para esfregar o chão e
outras receberam deveres diversos que completavam o trabalho por fazer.
Não são as atendentes que mantêm a instituição para as pobres pacientes,
como eu sempre imaginara, mas as próprias pacientes que trabalham por
si, até mesmo limpando os quartos e cuidando das roupas das enfermeiras.

Às nove e meia, as novas pacientes, das quais eu era uma, foram mandadas
para ver o doutor. Fui levada e meus pulmões e coração foram examinados
pelo médico jovem e namorador que foi o primeiro a nos ver no dia que
chegamos. O autor do relatório, se não me engano, foi o dr. Ingram, o
superintendente assistente. Depois de algumas perguntas, recebi
permissão para voltar à sala de estar.

Cheguei e vi a Srta. Grady com o meu caderno e lápis de grafite,
comprados especialmente para a ocasião.

--- Quero meu caderno e meu lápis --- disse honestamente. --- Eles me
ajudam a lembrar das coisas.

Eu estava muito ansiosa para fazer anotações e fiquei decepcionada com a
resposta:

--- Você não pode ficar com eles, então cale a boca.

Alguns dias depois, perguntei ao dr. Ingram se podia tê-los de volta e
ele prometeu considerar a questão. Quando toquei no assunto de novo, ele
respondeu que a Srta. Grady dissera que eu havia trazido apenas um
caderno e que não tinha um lápis. Irritada, insisti que tinha sim, ao
que fui aconselhada a lutar contra os devaneios que meu cérebro criava.

Depois que as pacientes terminaram o trabalho doméstico, como o dia
estava bonito, apesar de frio, recebemos a ordem de ir ao salão e buscar
nossos xales e chapéus para uma caminhada. Pobres pacientes! Como elas
ansiavam pelo ar fresco, como ansiavam por uma fuga mínima que fosse
daquela prisão. Todas foram correndo para o corredor e começaram a
brigar pelos chapéus. E que chapéus!

\imagemmedia{}{./media/rId54.png}

\label{section-6}

\chapter{Passeando com as
lunáticas}\label{capuxedtulo-xii-passeando-com-as-lunuxe1ticas}

Nunca vou esquecer minha primeira caminhada. Quando todas as pacientes
colocaram os chapéus de palha brancos, como aqueles que os banhistas
usam em Coney Island. Tive que rir da aparência cômica produzida. Era
impossível dizer quem era quem. Perdi a Srta. Neville e precisei tirar
meu chapéu para procurá-la. Quando nos encontramos, pusemos nossos
chapéus de volta e rimos uma da outra. Formamos uma fila aos pares e,
guardadas pelas atendentes, saímos pelo fundo do hospício para caminhar
pelas trilhas.

Depois de alguns metros, vi que de cada trilha saíam longas fileiras de
mulheres, sempre guardadas por enfermeiras. Tantas! Para todos os lados
que olhava, lá estavam elas, com seus vestidos estranhos, xales e
chapéus de palha engraçados, marchando lentamente. Fiquei fascinada por
aquelas filas e tremi com aquela visão terrível. Os olhos eram vazios,
os rostos insensatos, as línguas apenas repetiam baboseiras sem sentido.
Um grupo de pacientes passou por mim e além da visão, o olfato me
informou elas estavam imundas.

--- Quem são elas? --- perguntei a uma paciente ao meu lado.

--- Elas são consideradas as mais violentas da ilha. Elas ficam na
Cabana, o primeiro edifício, com os degraus altos. @ Algumas gritavam,
outras soltavam palavrões, outras cantavam ou rezavam ou pregavam,
dependendo do que lhes dava vontade, e juntas eram o exemplo de
humanidade mais miserável que jamais vira na vida. Quando a algazarra
daquele grupo se afastou, vi outra cena que nunca esquecerei:

Uma corda comprida estava afixada a cintos de couro largos, que por sua
vez ficavam presos em torno das cinturas de cinquenta e duas mulheres. A
ponta da corta estava amarrada a um carrinho de ferro pesado que levava
duas mulheres: uma com um pé machucado, outra que gritava para uma
enfermeira. @--- Você me bateu, nunca vou me esquecer disso! Você quer
me matar --- ela dizia, e então chorava e soluçava. @ As mulheres ``na
corda'', como as pacientes chamam, estavam todas ocupadas com suas
loucuras individuais. Algumas gritavam o tempo inteiro. Uma, de olhos
azuis, viu que eu estava a observando e se virou o quanto conseguiu,
falando e sorrindo, com aquele olhar aterrorador de insanidade absoluta
estampado no rosto. Os doutores não teriam dificuldade em decidir seu
caso. Para alguém que nunca estivera perto de uma pessoa insana antes, o
horror daquele olhar era inominável.

--- Deus as acuda! --- sussurrou a Srta. Neville. --- É terrível, não
consigo nem olhar.

Elas foram passando, com seus lugares sendo preenchidos por outras
tantas. Você consegue imaginar a cena? De acordo com um dos doutores,
havia 1.600 mulheres insanas na Ilha de Blackwell.

Louca! O que pode ser pior? Meu coração se derretia de piedade quando eu
olhava para aquelas senhoras grisalhas falando com o vazio. Uma mulher
vestia uma camisa de força e duas outras precisavam arrastá-la.
Aleijadas, cegas, velhas, jovens, feias e bonitas, uma grande massa
insensata de humanidade. Não existe destino pior.

Olhei para o lindo gramado, que antes imaginei ser um conforto para as
pobres criaturas confinadas à ilha, e ri de minhas próprias ideias. Que
prazer elas tinham? As pacientes eram proibidas de pisar na grama, só
podiam olhá-la. Vi algumas pacientes erguerem com carinho e admiração
uma noz ou uma folha colorida que havia caído na trilha, mas ninguém
tinha a permissão de guardá-las. As enfermeiras sempre as forçavam a
atirar fora o pedacinho de conforto que Deus lhes dava.

\imagemmedia{}{./media/rId57.png}Quando
passei por um pavilhão baixo, onde inúmeras lunáticas indefesas ficavam
confinadas, li o seguinte lema escrito na parede: ``Enquanto vivo, tenho
esperança''. O absurdo daquilo me deixou admirada. Senti vontade de
colocar um aviso sobre os portões do hospício: ``Quem aqui entra
abandona toda esperança''.

Durante a caminhada, fui bastante incomodada por enfermeiras que haviam
escutado minha história romântica. Elas chamavam as enfermeiras
encarregadas de nós e pediam que me apontassem, o que aconteceu diversas
vezes.

Não demorou para chegar a hora do almoço e minha fome era tamanha que eu
me sentia capaz de comer qualquer coisa. A velha história de esperar de
pé no corredor por meia hora ou quarenta minutos se repetiu antes que
recebêssemos nossa refeição. Desta vez, as tigelas em que tomávamos chá
continham sopa, enquanto o prato tinha uma batata cozida fria e um
pedaço de carne que, depois de uma breve investigação, se revelou
ligeiramente estragado. Não tínhamos garfos ou facas e as pacientes
pareciam selvagens, agarrando a carne dura com os dedos, mordendo e
puxando para arrancar pedaços. As que não tinham dentes, ou os tinham em
mau estado, não conseguiam comer. Ganhamos uma colher de sopa e um
pedaço de pão para completar o almoço. Nunca há manteiga no almoço, nem
café ou chá. A Srta. Mayard não conseguia comer, e vi muitas das doentes
se virarem da mesa com nojo. Eu estava ficando fraca por falta de comida
e experimentei uma fatia de pão. Depois das primeiras mordidas, a fome
se fez sentir e eu consegui comer quase tudo, deixando apenas a casca da
fatia.

O Superintendente Dent atravessou a sala de estar, distribuindo um
``como vai você?'' ou ``tudo bem?'' ocasional entre as pacientes. Sua
voz era tão fria quanto o corredor e as pacientes não tentaram contar
seus sofrimentos. Pedi a algumas que contassem sobre como sofriam por
causa do frio e da escassez de roupas, mas elas responderam que a
enfermeira as espancaria se falassem alguma coisa.

Nunca fiquei tão cansada quanto sentada naqueles bancos. Várias
pacientes se sentavam sobre um pé ou de lado para variar sua posição,
mas sempre eram repreendidas e precisavam retesar. Se conversavam, eram
xingadas e lhes mandavam calar a boca; se queriam caminhar um pouco para
relaxar o corpo, lhes mandavam sentar e ficar paradas. O que, além da
tortura, produziria insanidade com mais rapidez do que esse tratamento?
Aqui temos um grupo de mulheres que deveriam ser curadas. Gostaria de
ver os médicos especialistas que me condenam por minhas ações, que
comprovaram suas capacidades, escolherem uma mulher perfeitamente sã e
saudável, trancarem-na e forçarem-na a sentar-se das seis da manhã às
oito da noite em bancos de costas retas, não permitirem que conversasse
ou se movesse durante essas horas, não lhe dessem material de leitura
nem lhe informassem sobre os acontecimentos do mundo e fornecessem
comida ruim e tratamento ruim e então vissem quanto demoraria para que
ela enlouquecesse. Dois meses bastariam para deixá-la física e
mentalmente destroçada.

Descrevi meu primeiro dia no hospício, mas como meus outros nove foram,
em linhas gerais, exatamente iguais, seria tedioso recontar cada um. Ao
escrever esta história, espero que muitas das pessoas desmascaradas
tentem me contradizer. Estou apenas contando em linguagem simples, sem
exageros, minha vida em um hospício por dez dias. A alimentação foi uma
das coisas mais horríveis. Exceto pelos dois primeiros dias depois de
minha chegada, não havia sal em nada. As mulheres famintas tentavam
comer aqueles pratos horríveis. Colocava-se mostarda e vinagre na carne
e na sopa para dar algum gosto, mas o resultado era apenas pior. Depois
de dois dias, mesmo isso tinha se esgotado, e as pacientes precisavam
tentar engolir peixe fresco, recém-fervido em água, sem sal, pimenta ou
manteiga; carne de cordeiro e de gado e batatas sem o menor tempero. As
mais insanas se recusavam a engolir a comida e eram ameaçadas com
castigos. Em nossas breves caminhadas, passávamos pela cozinha onde se
preparavam as refeições para as enfermeiras e os doutores. Lá, víamos
melões e uvas e frutas de todos os tipos, pães brancos lindos e carnes
de qualidade, e nossa fome se multiplicava por dez. Falei com alguns dos
médicos, mas sem resultado, e quando fui solta a comida ainda não tinha
nada de sal.

Eu sentia uma dor no coração ao ver as pacientes adoecerem cada vez mais
sobre a mesa. Vi a Srta. Tillie Mayard sofrer um mal súbito depois de
morder a comida e precisar sair correndo da sala de jantar, apenas para
ser repreendida por si. Quando as pacientes reclamavam da comida, a
resposta era que calassem a boca; que em casa sua vida não seria tão boa
e que aquilo era tudo bom demais para pacientes recebendo caridade.

Uma menina alemã chamada Louise (esqueço seu sobrenome) não comeu por
vários dias e, uma certa manhã, não apareceu. Pela conversa das
enfermeiras, descobri que estava febril. A pobrezinha me contara que
rezava sem descanso pela morte. Observei as enfermeiras forçarem uma
paciente a levar a comida recusada pelas saudáveis até o quarto de
Louise. Imagine, dar aquilo para uma paciente com febre! Obviamente, ela
recusou. Foi então que vi uma enfermeira, a Srta. McCarten, medir sua
temperatura e voltar dizendo que era de cerca de 65 graus. Sorri com a
informação e a Srta. Grupe, vendo minha reação, perguntou qual fora
minha maior temperatura na vida. Não respondi. A Srta. Grady decidiu
testar suas habilidades e voltou informando uma febre de 37,2 graus.

A Srta. Tillie Mayard era quem mais sofria com o frio, mas ainda tentava
seguir meu conselho de se animar e tentar se manter ativa por algum
tempo. O Superintendente Dent trouxe um homem para me ver. Ele mediu meu
pulso e examinou minha cabeça e minha língua. Informei os dois sobre o
frio e garanti que não precisava de atenção médica, ao contrário da
Srta. Mayard, e pedi que transferissem sua atenção para ela. Os dois não
responderam, mas fiquei feliz em ver a Srta. Mayard se erguer do banco e
se aproximar deles. Ela falou com os doutores e contou que estava
doente, mas eles não prestaram atenção. As enfermeiras apareceram e a
arrastaram de volta para o banco. @--- Depois de um tempo, quando vir
que os médicos não dão bola para você, vai parar de correr para eles ---
elas disseram depois que os médicos foram embora. @ Antes de partirem,
ouvi um dos médicos dizer (não lembro suas palavras exatas) que meu
pulso e meus olhos não eram os de uma menina insana, mas o
Superintendente Dent lhe garantiu que em casos como o meu, esse tipo de
teste não funciona. Depois de me observar por algum tempo, ele disse que
meu rosto era o mais inteligente que já vira em uma lunática. As
enfermeiras vestiam casacos e roupas de baixo pesadas, mas se recusavam
a nos entregar os xales.

Passei a madrugada inteira ouvindo uma mulher chorar de frio e pedir a
Deus que a deixasse morrer. Outra repetia o grito de ``Assassino!'', ou
então ``Polícia!'', até minha pele ficar arrepiada.

Na segunda manhã, depois de começarmos nossa rotina interminável do dia,
duas das enfermeiras, auxiliadas por algumas pacientes, trouxeram a
mulher que passara a madrugada anterior implorando que Deus a levasse.
Não fiquei surpresa com a oração. Ela parecia ter no mínimo setenta anos
de idade e era cega. Apesar dos corredores estarem absolutamente
gelados, aquela senhora não tinha mais roupa do que qualquer uma de nós.

--- Ah, o que estão fazendo comigo? --- ela gritou quando foi levada
para a sala de estar e colocada sobre o banco duro. --- Por que não
posso ficar na cama, por que não me dão um xale? @ A seguir, ela se
levantava e saía tateando pela sala em busca da saída. Às vezes, as
atendentes a puxavam de volta para o banco; às vezes, deixavam ela
caminhar, rindo maldosamente quando ela esbarrava contra mesa ou na
ponta de um banco. Em uma ocasião, ela disse que os sapatos pesados
fornecidos pela caridade machucavam seus pés e então tirou-os. As
enfermeiras mandaram que duas pacientes recolocassem os sapatos; depois
que ela os tirou várias vezes, e brigou para não os colocar de volta,
contei sete pessoas tentando calçá-la. Depois disso, a senhora tentou se
deitar no banco, mas elas a puxaram de novo. Dava uma pena terrível
ouvir ela gritar e chorar:

--- Ai, alguém me dê um travesseiro e me tape com o cobertor. Estou com
frio, muito frio.

Foi então que vi a Srta. Grupe sentar-se sobre a senhora e passar suas
mãos geladas no rosto da mulher e dentro da gola dos vestidos. Quando a
velha senhora chorou, a enfermeira soltou uma gargalhada selvagem, assim
como as outras enfermeiras, e repetiu sua crueldade. Naquele mesmo dia,
a senhora foi levada para outra ala.

\label{section-7}

\chapter{Sufocando e espancando
pacientes}\label{capuxedtulo-xiii-sufocando-e-espancando-pacientes}

A Srta. Tillie Mayard sofria enormemente com o frio. Uma manhã, ela
sentou-se no banco ao meu lado, pálida de frio. Seus braços e pernas
tremiam, seus dentes tiritavam. Falei com as três atendentes, que
estavam sentadas na mesa central usando casacos.

--- É cruel trancar pessoas e deixá-las congelar --- eu disse. @ A
resposta foi que ela tinha tanto quanto as outras e não receberia mais.
Foi quando a Srta. Mayard teve um acesso que deixou todas as pacientes
assustadas. A Srta. Neville a segurou e abraçou, mas tudo que as
enfermeiras disseram foi:

--- Deixa ela cair, assim aprende a lição. @ A Srta. Neville respondeu
com o que pensava sobre suas ações e eu recebi a ordem de me dirigir ao
escritório.

Quando cheguei, o Superintendente Dent apareceu na porta e eu lhe contei
sobre o sofrimento causado pelo frio e sobre a condição da Srta. Mayard.
Não duvido que minha fala tenha sido incoerente, pois também falei sobre
o estado da comida, o tratamento dado pelas enfermeiras, sua recusa em
distribuir mais roupas, a condição da Srta. Mayard e como as enfermeiras
nos diziam que, como o hospício era uma instituição pública, nem por
bondade poderíamos esperar. Garantindo que eu própria não precisava de
auxílio médico, pedi que ele fosse ver a Srta. Mayard. Foi o que ele
fez. A Srta. Neville e outras pacientes me contaram o que aconteceu. A
Srta. Mayard ainda estava tendo um acesso, então ele a agarrou com força
pela testa e apertou até seu rosto se avermelhar com o acúmulo de sangue
na cabeça e ela se recuperar. Pelo resto do dia, ela sofreu de uma dor
de cabeça terrível, e depois disso apenas piorou.

Insana? Sim, insana. Assistindo a insanidade se infiltrar em uma mente
que parecera estar bem, amaldiçoei secretamente os médicos, as
enfermeiras e todas as instituições públicas. Alguns dirão que ela
enlouquecera em algum momento antes de ser confinada no hospício. Se
estão certos, aquele foi o lugar adequado para uma mulher em
convalescença? Um lugar onde recebeu apenas banhos gelados, roupa
insuficiente e alimentação horrível?

Naquela manhã, tive uma longa conversa com o dr. Ingram, o
superintendente assistente do hospício. Descobri que ele era bondoso com
as infelizes. Comecei com minha velha reclamação sobre o frio, então ele
chamou a Srta. Grady para o escritório e mandou que mais roupas fossem
distribuídas para os pacientes. A Srta. Grady me avisou que se
continuasse a ser dedo-duro, as consequências seriam graves.

Muitos visitantes em busca de meninas desaparecidas vieram me ver. Um
dia, a Srta. Grady gritou da porta do corredor:

--- Nellie Brown, estão chamando você.

Fui até a sala de estar no final do corredor, onde encontrei um
cavalheiro que me conhecia intimamente havia muitos anos. Pela palidez
súbita em seu rosto e sua incapacidade de falar, percebi que minha
presença era totalmente inesperada e havia causado um choque terrível
nele. No mesmo instante, resolvi que se me desmascarasse como Nellie
Bly, eu diria que nunca o vira antes. Contudo, ainda tinha uma carta na
manga e decidi arriscar. Com a Srta. Grady tão perto, sussurrei para ele
apressada, usando um linguajar mais expressivo do que elegante:

--- Não me entregue.

Pela expressão em seus olhos, percebi que ele entendera, então disse
para a Srta. Grady:

--- Não conheço este homem.

--- Você conhece ela? --- a Srta. Grady perguntou.

--- Não, essa não é a jovem que estou procurando --- ele respondeu com
uma voz forçada.

--- Se não conhece ela, não pode ficar aqui --- a enfermeira disse,
levando-o para a porta. @ De repente, temi que ele achasse que eu fora
enviada ao hospício por engano e que contaria aos meus amigos e tentaria
fazer com que me soltassem. Assim, esperei até a Srta. Grady destrancar
a porta. Eu sabia que ela precisaria trancá-la de volta antes de poder
sair, então o tempo necessário para a operação me daria a oportunidade
de falar com meu amigo.

--- Um instante, \#\#señor\textbackslash{}\#\# --- chamei. @ Ele voltou
e perguntei em voz alta:

--- O senhor fala espanhol? --- e então continuei cochichando: --- Está
tudo bem, estou atrás de um item. Fique quieto. @--- Não --- ele disse,
com uma ênfase especial que significava que guardaria meu segredo.

As pessoas no mundo exterior não têm como imaginar o comprimento dos
dias para quem está em um hospício. Eles parecem eternos, então
recebemos de braços abertos qualquer evento que nos dê algo para pensar
ou conversar. Não há nada para ler e o único assunto que nunca cansa é
imaginar os alimentos refinados que comeremos assim que sairmos. Todas
aguardam ansiosamente a chegada do barco, querendo ver se alguma nova
infeliz será adicionada ao grupo. Quando chegam e são levadas à sala de
estar, as pacientes se solidarizam umas com as outras por ela e ficam
ansiosas para demonstrar sua atenção. O Salão 6 era o setor de recepção,
então era assim que víamos todas as recém-chegadas.

Logo depois de minha chegada, uma menina chamada Urena Little-Page foi
trazida. Ela era tola desde nascença e seu ponto fraco era, assim como
ocorre com muitas mulheres sensatas, sua idade. Ela afirmava ter dezoito
anos e ficava furiosa quando lhe diziam o contrário. As enfermeiras não
demoraram para descobrir isso e gostavam de provocá-la.

--- Urena --- a Srta. Grady dizia. --- Os médicos dizem que você tem
trinta e três, não dezoito. @ E as outras enfermeiras riam. Elas
continuaram com isso até aquela criatura simplória começar a berrar e a
chorar, dizendo que queria voltar para casa e que todas a tratavam mal.
Depois de se divertirem tudo que queriam e deixarem-na chorando,
começaram a repreendê-la e mandaram que calasse a boca. A menina foi
ficando cada vez mais histérica, até que as enfermeiras saltaram sobre
ela, estapearam seu rosto e bateram sua cabeça com toda força. Isso fez
com que a pobre mulher chorasse ainda mais, então elas a esganaram. Sim,
esganaram de verdade. Depois ela foi arrastada até um gabinete e ouvi os
gritos aterrorizados se transformarem em sons abafados. Depois de várias
horas de ausência, ela voltou à sala de estar. As marcas dos dedos ao
redor da garganta ficaram evidentes pelo resto do dia.

Esse castigo pareceu acordar nelas o desejo de administrar mais e mais.
Elas voltaram à sala de estar e agarraram uma senhora grisalha, uma
mulher que já vi chamada de sra. Grady e de sra. O'Keefe. Ela era insana
e falava quase continuamente, tanto sozinha quanto com aquelas ao seu
redor. Ela nunca falava muito alto, e na ocasião a que me refiro estava
apenas sentada, tagarelando consigo mesmo sem incomodar ninguém. As
enfermeiras a agarraram e seus gritos me deram uma pontada no coração:

--- Pelo amor de Deus, senhoras, não deixem que me batam!

--- Cala a boca, sua abusada! --- a Srta. Grady exclamou, agarrando a
mulher pelo cabelo grisalho. @ A enfermeira arrastou a mulher para
longe, gritando e implorando. A senhora também foi levada ao gabinete.
Seus gritos foram ficando mais e mais baixos, até sumirem.

As enfermeiras voltaram ao quarto e a Srta. Grady afirmou que havia
``dado um jeito na velha idiota por um tempo''. Contei a alguns dos
médicos sobre o ocorrido, mas eles não me deram atenção.

Uma das personagens do Salão 6 era Matilda, uma velhinha alemã que,
creio, enlouqueceu depois de perder seu dinheiro. Ela era baixinha e
tinha uma pele rosada muito bonita. Ela não incomodava muito, exceto em
algumas ocasiões. Ela tinha acessos em que conversava com os aquecedores
ou então subia em uma cadeira para falar pela janela. Nessas conversas,
ela ralhava com os advogados que haviam roubado sua propriedade. As
enfermeiras pareciam se divertir muito provocando aquela senhora
inofensiva. Um dia, sentada ao lado da Srta. Grady e da Srta. Grupe,
ouvi elas lhe dizerem coisas absolutamente horríveis para chamar a Srta.
McCarten. Depois de mandá-la dizer essas coisas, elas chamaram a outra
enfermeira, mas Matilda mostrou que, mesmo naquele estado, tinha mais
bom senso do que as atendentes.

--- Não posso contar. É particular --- era tudo que dizia. @ Eu vi a
Srta. Grady, fingindo cochichar algo para ela, cuspir na orelha da
mulher. Matilda limpou a orelha em silêncio e não disse nada.

\label{section-8}

\chapter{Algumas histórias
infelizes}\label{capuxedtulo-xiv-algumas-histuxf3rias-infelizes}

A essa altura, eu já havia conhecido boa parte das quarenta e cinco
mulheres no Salão 6. Gostaria de apresentar algumas. Louise, a alemã
bonitinha que mencionei ter sofrido uma febre, acreditava ser
acompanhada pelos espíritos dos pais mortos. @--- Levei várias surras da
Srta. Grady e de suas assistentes --- ela contou. --- E não consigo
engolir essa comida horrível que nos dão. Eu não deveria ficar
congelando por falta de roupas decentes. Ah! Todas as noites, rezo para
ser levada até papai e mamãe. Uma noite, quando estava confinada em
Bellevue, o dr. Field apareceu. Eu estava na cama, esgotada pelos
exames. Finalmente, eu disse: ``Cansei disso, não vou falar mais''.
``Não vai, é?'', ele respondeu, brabo. ``Vamos ver se não fazemos você
falar''. Com isso, ele colocou a bengala ao lado da cama, subiu nela e
me deu beliscões nas costelas. Eu dei um salto na cama e disse: ``Mas o
que você está fazendo?'' Ele respondeu: ``Estou ensinando você a
obedecer''. Se ao menos eu pudesse morrer e voltar para papai! @ Quando
fui solta, ela estava confinada no leito, febril, e talvez seu desejo já
tenha sido atendido.

No Salão 6 há uma francesa, ou havia durante a minha estadia, que tenho
certeza absoluta ser perfeitamente sã. Observei-a e conversei com ela
todos os dias, exceto nos últimos três, e não encontrei nenhum sinal de
mania ou devaneio nela. Seu nome é Josephine Despreau, se acertei na
ortografia, e seu marido e todos os seus amigos estão na França.
Josephine sofre muito com sua situação. Seus lábios tremem e ela tem
ataques de choro quando fala sobre sua condição. @--- Como você veio
parar aqui? --- perguntei.

--- Uma manhã, quando estava tentando tomar café da manhã, fiquei
muitíssimo doente. A dona da casa chamou dois policiais e eu fui levada
para a delegacia. Não consegui entender o que estava acontecendo e eles
não deram atenção para a minha a história. O modo como as coisas
funcionam neste país ainda era novidade para mim. Antes que me desse por
conta, fui presa neste hospício, dada como louca. Quando cheguei,
comecei a chorar por estar aqui sem nenhuma chance de sair. Por causa do
choro, a Srta. Grady e suas assistentes me esganaram até machucar minha
garganta, e ela dói desde então.

Uma menina hebraica, jovem e bonitinha, falava tão pouco inglês que só
consegui obter sua história através das enfermeiras. Elas disseram que
seu nome era Sarah Fishbaum e que seu marido a internara no hospício
porque ela tinha uma paixão por homens que não ele próprio. Se
concedermos que Sarah era insana, e que sua loucura era por homens,
gostaria de contar como as enfermeiras tentavam ``curá-la''. Elas a
chamavam e diziam:

--- Sarah, você não queria ter um jovem simpático?

--- Oh, sim, um jovem é bom --- Sarah respondia com as poucas palavras
que sabia do nosso idioma.

--- Ora, Sarah, não quer que falemos de você para alguns dos doutores?
Você não gostaria de ficar com um dos doutores?

E assim elas faziam, perguntando qual médico ela preferia,
aconselhando-a a paquerá-lo quando ele visitasse o salão e assim por
diante.

Depois de alguns dias observando e conversando com uma mulher de pele
clara, não conseguia sequer imaginar por que ela fora mandada para o
hospício, de tão sã que era.

--- Por que você veio para cá? --- eu perguntei um dia, depois de uma
longa conversa com ela.

--- Eu estava doente --- ela respondeu.

--- Você tem uma doença mental? --- eu insisti.

--- Não, não, de onde você tirou essa ideia? Eu estava trabalhando
demais e acabei ficando debilitada. Por ter problemas na família, e por
não ter um centavo no bolso e nem para onde ir, solicitei aos
comissários que fosse mandada para um asilo de indigentes até conseguir
voltar ao trabalho.

--- Mas eles não mandam os pobres para cá a menos que sejam loucos ---
respondi. --- Você não sabia que apenas mulheres insanas, ou
supostamente insanas, são mandadas para cá?

--- Depois de chegar, percebi que quase todas essas mulheres eram
loucas, mas na época acreditei quando me disseram que esse era o lugar
para onde mandavam a gente pobre que pedia socorro como eu fizera.

--- Como têm lhe tratado? --- perguntei. @--- Bem, por ora, me escapei
de ser espancada, mas fico doente de ver o que acontece com as outras e
de ouvir as histórias. Quando fui trazida, elas foram me dar banho, mas
a doença para a qual precisava de tratamento e da qual estava sofrendo
me obrigava a não tomar banhos. Ainda assim elas me colocaram no banho,
e meu sofrimento aumentou muito por várias semanas.

Uma sra. McCartney, cujo marido é alfaiate, parece perfeitamente
racional e não é dada a fantasia alguma. Mary Hughes e a sra. Louise
Schanz não demonstraram nenhum sinal óbvio de insanidade.

Um dia, duas recém-chegadas foram adicionadas ao nosso grupo. A primeira
era uma idiota chamada Carrie Glass, a segunda uma menina alemã
simpática que parecia muito jovem. Quando entrou, todas as pacientes
comentaram sobre sua aparência e sobre sua óbvia sanidade. Seu nome era
Margaret. Ela me contou que fora cozinheira e que era extremamente
ordeira. Um dia, depois de ter esfregado o chão da cozinha, as
camareiras desceram e o sujaram propositalmente. Ela teve um acesso de
raiva e começou a brigar com elas; um policial foi chamado e ela foi
levada ao hospício.

--- Como é que podem dizer que sou louca? Só porque não me segurei uma
vez? --- ela reclamou. --- Outras pessoas não são trancadas e chamadas
de loucas quando se irritam. Tudo que eu posso fazer é ficar quieta e
evitar as surras que vejo as outras levarem. Ninguém pode dizer nada
sobre mim. Faço tudo o que me mandam, todo o trabalho que me dão. Sou
obediente em tudo e faço tudo para provar para eles que sou sã.

Um dia, uma mulher insana foi trazida. Ela era barulhenta, então a Srta.
Grady lhe espancou e a deixou com o olho roxo. Quando os médicos
perceberam e perguntaram se aquilo acontecera antes dela ser trazida, as
enfermeiras responderam que sim.

Enquanto estava no Salão 6, nunca ouvi as enfermeiras falarem com as
pacientes que não fosse para repreendê-las ou gritar com elas, a menos
que fosse para provocá-las. Elas passavam quase todo o tempo fofocando
sobre os médicos e sobre as outras enfermeiras em um estilo nada
edificante. A Srta. Grady quase sempre pontuava sua conversa com
profanidades e gostava de começar suas frases invocando o nome do
Senhor. Os nomes que dava às pacientes eram do tipo mais vil e profano.
Uma noite, durante nosso jantar, ela discutiu com outra enfermeira por
causa do pão; quando a outra se afastou, ela a chamou de coisas
horríveis e fez observações maldosas sobre ela.

À noite, uma mulher, que imagino ser a cozinheira-chefe para os
doutores, costumava aparecer com passas, uvas, maçãs e bolachas salgadas
para as enfermeiras. Imagine a sensação das pacientes famintas,
assistindo as enfermeiras comerem algo que para elas seria um luxo
inalcançável.

Uma tarde, o dr. Dent conversava com uma paciente, a sra. Turney, sobre
um problema que esta tivera com uma enfermeira ou governanta. Logo
depois que fomos levadas para o jantar e a mulher que batera na sra.
Turney, e da qual o dr. Dent falara, estava sentada junto à porta de
nossa sala de jantar. De repente, a sra. Turney pegou sua tigela de chá,
saiu correndo pela porta e jogou tudo na mulher que a espancara.
Ouviram-se berros e a sra. Turney foi levada de volta ao seu lugar. No
dia seguinte, ela foi transferida para a ``corda'', supostamente
composta das mulheres mais perigosas e suicidas da ilha.

No início eu não conseguia dormir, e nem queria enquanto conseguisse
ouvir alguma novidade. As enfermeiras noturnas podem ter reclamado
disso. Uma noite, elas entraram e tentaram me forçar a tomar um copo com
uma mistura para ``me fazer dormir'', segundo disseram. Respondi que não
queria nada do tipo e elas me deixaram em paz, ou assim esperava, pelo
resto da noite. Minhas esperanças foram em vão, pois em alguns minutos
elas voltaram com um doutor, o mesmo que nos recebera na chegada. Ele
insistiu que eu tomasse a dose, mas eu estava decidida a me manter
sempre desperta, sem descansar mesmo por algumas horas. Quando ele viu
que seria impossível me convencer, seu comportamento se embruteceu e ele
disse que já tinha desperdiçado tempo demais comigo, que se não tomasse
a bebida ela seria injetada em meu braço com uma agulha. Percebi que se
ele me injetasse no braço, seria impossível me livrar da mistura, mas se
engolisse eu teria alguma chance, então concordei. Ela cheirava a
láudano, e a dose era horrível. Assim que eles saíram do quarto e me
trancaram lá dentro, tentei descobrir até onde conseguia enfiar meu dedo
na garganta e o cloral pôde tentar colocar o resto do quarto para
dormir.

Gostaria de dizer que Burns, a enfermeira noturna do Salão 6, parecia
ser muito bondosa e paciente com as pobres infelizes sob sua guarda. As
outras enfermeiras fizeram várias tentativas de conversar comigo sobre
namorados e me perguntaram se eu gostaria de ter um. Não fui muito
comunicativa sobre esse assunto tão popular entre elas.

Uma vez por semana, as pacientes recebem um banho, e esta é a única vez
que veem sabão. Um dia, uma paciente me entregou um pedaço de sabão do
tamanho de um dedal, o que considerei um grande elogio. Ela queria ser
bondosa, mas achei que ela aproveitaria o sabão barato melhor do que eu,
então agradeci, mas não aceitei. Durante o banho, enche-se a banheira e
as pacientes são lavadas, uma após a outra, sem trocarem a água. Isso
continua até a água ficar bastante espessa, quando então a banheira é
esvazia e re-enchida, sem antes de ser lavada. As mesmas toalhas são
usadas em todas as mulheres, com ou sem erupções na pele. As pacientes
saudáveis lutam para que a água seja trocada, mas são forçadas a se
submeter às ordens das enfermeiras preguiçosas e tiranas. Os vestidos
raramente são trocados mais do que uma vez ao mês. Se a paciente tem um
visitante, é comum ver as enfermeiras correrem para trocar seu vestido
antes que o estranho entre, o que esconde a verdadeira administração do
lugar.

As pacientes que não conseguiam cuidar de si mesmas acabavam em uma
condição bestial. As enfermeiras as ajudavam, apenas mandavam que as
outras pacientes o fizessem.

Por cinco dias, fomos forçadas a passar o tempo inteiro sentadas. Nunca
o tempo passou tão lentamente. Todas as pacientes estavam tesas e
doloridas e cansadas. Em grupinhos, torturávamos nossos estômagos
conjurando ideias do que comeríamos primeiro quando saíssemos. Se não
soubesse a fome que elas passavam e o aspecto miserável da situação, a
conversa teria sido muito engraçada. Naquele lugar, só me deixava
triste. Quando o assunto da comida, que parecia ser o favorito, acabava,
elas opinavam sobre a instituição e sua administração. A condenação das
enfermeiras e dos alimentos era unânime.

Com o passar dos dias, a condição da Srta. Tillie Mayard foi piorando.
Ela estava sempre gelada e não conseguia ingerir a comida que lhe era
fornecida. Ela cantava todos os dias para tentar preservar sua memória,
mas a enfermeira forçou-a a parar. Eu conversava com ela todos os dias e
me entristecia assistir aquela deterioração tão rápida. Finalmente, ela
teve um devaneio. Ela achou que eu estava roubando o seu lugar e que
todas as pessoas que apareceram para visitar Nellie Brown eram amigos em
busca dela, mas que eu, de alguma forma, estava tentando enganá-los para
que acreditassem que eu era a própria. Tentei usar a lógica com ela, mas
era impossível, então me afastei para que minha presença não piorasse
seu estado e alimentasse sua loucura.

Um dia, uma das pacientes, a bela e delicada sra. Cotter, achou que vira
seu marido vindo pela trilha, então saiu da fila em que estava marchando
e correu para encontrá-lo. Por esse ato, ela foi mandada para o Retiro.

--- Basta lembrar daquilo para me deixar furiosa --- ela diria mais
tarde. --- Por chorar, as enfermeiras me bateram com um cabo de vassoura
e pularam em cima de mim. Sofri ferimentos internos que nunca vão sarar.
Depois, elas amarraram minhas mãos e meus pés, atiraram um pano sobre a
minha cabeça, torceram bem apertado ao redor da minha garganta para que
eu não conseguisse gritar e me atiraram em uma banheira cheia de água
gelada. Elas me seguraram embaixo d'água até eu desistir completamente e
desmaiar. Outras vezes, elas me agarravam pelas orelhas e batiam minha
cabeça contra o chão e contra a parede. Elas também arrancaram meu
cabelo pela raiz, então ele não tem como crescer de volta.

A sra. Cotter me mostrou as provas de sua história: a depressão na parte
de trás da cabeça e as falhas no couro cabeludo onde punhados de cabelo
haviam sido arrancados. Conto sua história com toda a simplicidade que
consigo: @--- Meu tratamento não é o pior que já vi por aqui, mas
arruinou minha saúde. Mesmo que consiga sair, vou estar destruída.
Quando meu marido ouviu falar do tratamento que recebi, ele ameaçou
desmascarar esse lugar se eu não fosse removida, então fui transferida
para cá. Mentalmente, hoje estou bem. Aquele medo todo passou e o doutor
prometeu que vão deixar meu marido me levar para casa.

Também conheci uma mulher chamada Bridget McGuinness, que no momento
parece ser sã. Ela contou que foi enviada ao Retiro 4 e colocada ``na
corda''. @--- A surra que levei lá foi uma coisa horrível. Eu era puxada
pelos cabelos, me seguravam embaixo d'água até afogar e me esganavam e
chutavam. As enfermeiras sempre deixavam uma paciente calma na janela
para avisar quando algum dos doutores estava se aproximando. Não
adiantava nada reclamar para os doutores, eles sempre diziam que era
imaginação dos nossos cérebros doentes, e ainda levávamos outra surra
por abrir a boca. Elas afogavam as pacientes e ameaçavam deixar elas lá
para morrer se não prometessem não contar nada para os doutores. Nós
sempre prometíamos, pois sabíamos que os doutores não iriam nos ajudar,
e faríamos de tudo para escapar dos castigos. Depois de quebrar uma
janela, fui transferida para a Cabana, o pior lugar da ilha. A sujeira
lá era terrível e o cheiro, pior ainda. No verão, o lugar fica cheio de
moscas. A comida é pior do que o que se vê nas outras alas e recebemos
apenas pratos de lata. Em vez das grades estarem no lado de fora, como
nesta ala, elas estão no lado de dentro. Muitas pacientes quietas estão
lá há anos, mas as enfermeiras seguram elas lá para fazer o trabalho. Em
uma das surras que levei lá, as enfermeiras pularam em cima de mim e
quebraram duas costelas.

--- Enquanto estava lá, trouxeram uma menina bonitinha. Ela estivera e
doente e brigou muito para não a levarem para aquele lugar suja. Uma das
enfermeiras noturnas pegou ela e deu uma surra, depois a atirou pelada
no banho frio e então jogou-a na cama. De manhã, a menina estava morta.
Os doutores disseram que ela morreu por causa das convulsões e esse foi
o fim da história.

--- Eles injetam tanta morfina e cloral que as pacientes enlouquecem. Já
vi pacientes berrando por água por causa do efeito das drogas e as
enfermeiras se recusam a dá-la. Já ouvi mulheres passarem a noite
inteira implorando por uma gota que fosse e não receberem. Eu mesma
fiquei gritando por água até minha boca ficar tão ressecada que não
conseguia falar.

Vi o mesmo no Salão 7 com meus próprios olhos. As pacientes imploravam
por algo para beber antes de deitarem, mas as enfermeiras (a Srta. Hart
e as outras) se recusavam a abrir o banheiro para que pudessem saciar
sua sede.

\label{section-9}

\chapter{Incidentes da vida no
hospício}\label{capuxedtulo-xv-incidentes-da-vida-no-hospuxedcio}

Nas alas, não há quase nada que ajude a passar o tempo. Todas as roupas
do hospício são feitas pelas pacientes, mas a costura não emprega a
mente. Depois de vários meses de confinamento, as ideias sobre o mundo
vão desaparecendo e tudo que as pobres prisioneiras têm para fazer é
sentar-se e refletir sobre seu destino. Nos salões superiores, as
pacientes têm uma boa vista dos barcos que passam e da cidade de Nova
Iorque. Muitas vezes, vendo as luzes distantes da cidade por entre as
grades, tentei imaginar como me sentiria se não tivesse ninguém para
obter minha soltura.

Vi muitas pacientes paradas, olhando cheias de saudade aquela cidade que
provavelmente nunca mais visitarão. Ela significa vida e liberdade; ela
parece tão próxima, mas o céu não está mais distante do inferno.

As mulheres têm saudade de casa? Exceto pelos casos mais violentos, elas
estão cientes de estarem confinadas em um hospício. O único desejo que
nunca morre é o da liberdade, do lar.

--- Sonhei com minha mãe ontem à noite --- uma pobre menina me dizia
todas as manhãs. --- Acho que ela vai vir hoje para me levar para casa.
@ Essa única ideia, esse anseio, está sempre presente, apesar dela estar
confinada há quatro anos.

Como é misteriosa a loucura. No hospício, vi pacientes cujos lábios
estão travados em um silêncio perpétuo. Elas vivem, respiram, comem; a
forma humana está ali, mas algo está ausente, algo sem o qual o corpo
sobrevive, mas que não existe sem o corpo. Muitas vezes me perguntei se
aqueles lábios escondiam sonhos que ficam além de nosso alcance ou
apenas o vazio.

Igualmente tristes são os casos em que as pacientes estão sempre falando
com interlocutores invisíveis. Vi muitas delas sem nenhuma consciência
sobre seu ambiente, absortas na conversa com um ser invisível. Contudo,
por mais estranho que seja, qualquer ordem que recebam é sempre
obedecida, tal e qual um cão obedece seu mestre. Uma das ilusões mais
patéticas que vi entre as pacientes foi a de uma menina irlandesa de
olhos azuis que acreditava estar condenada para todo o sempre por causa
de um ato em sua vida. @--- Estou danada por toda a eternidade! --- ela
urrava, dia e noite. @ Aquele grito enchia minha alma de horror e sua
agonia parecia uma visão do inferno.

Depois de transferida para o Salão 7, passei todas as noites trancada em
um quarto com seis loucas. Duas pareciam nunca dormir e passavam a noite
inteira delirando. Uma se levantava da cama e se arrastava pelo quarto
em busca de alguém que queria matar. Era inevitável considerar a
facilidade com a qual poderia atacar qualquer uma das outras pacientes
confinadas com ela, o que não tornava as noites mais confortáveis.

Uma mulher de meia-idade, que costumava sentar-se sempre em um dos
cantos da sala, ficou estranhamente afetada. Ela tinha um pedaço de
jornal do qual lia o tempo inteiro as histórias mais maravilhosas que já
ouvi na vida. Eu costumava me sentar ao seu lado para escutá-la. A
história e o romance fluíam igualmente bem de seus lábios.

Vi apenas uma carta ser entregue a uma paciente enquanto estava lá,
gerando muito interesse. Todas as pacientes pareciam sedentas por
notícias do mundo exterior. Elas se reuniram em torno da que tivera
aquela sorte e fizeram centenas de perguntas.

As visitas geram interesse e muita alegria. A Srta. Mattie Morgan, do
Salão 7, entreteve alguns visitantes um dia ao piano. Eles estavam perto
dela até alguém cochicharam que era uma paciente. @--- Louca! --- eles
sussurram audivelmente, afastando-se e deixando-a sozinha. @ Ela achou o
episódio engraçado, mas também ficou indignada. A Srta. Mattie,
auxiliada por diversas meninas que treinou, faz com que as tardes sejam
mais agradáveis no Salão 7. Elas cantam e dançam, muitas vezes os
doutores aparecem e dançam com as pacientes.

Um dia, quando fomos almoçar, ouvimos um chorinho vindo do porão. Todas
pareciam notá-lo e não demorou até descobrirmos que havia um bebê lá
embaixo. Sim, um bebê. Pense só: um bebezinho inocente, nascido naquela
câmara de horrores! Não consigo imaginar nada mais terrível.

Uma visitante apareceu um dia trazendo um bebê nos braços. Uma mulher
que fora separada dos cinco filhos pequenos pediu permissão para
segurá-lo. Quando a visitante quis ir embora, a tristeza da mulher foi
incontrolável. Ela implorava para ficar com o bebê, que imaginava ser um
de seus filhos. A cena inflamou mais pacientes do que eu já havia visto
até então.

A única diversão ao ar livre dos pacientes, se é que merece esse nome,
quando o tempo permite, é andar no carrossel uma vez por semana. É uma
mudança na rotina, então elas demonstram um certo prazer com a
oportunidade.

As pacientes mais dóceis trabalham em uma fábrica de esfregões, uma
fábrica de tapetes e na lavanderia. Em troca, não recebem recompensa
alguma além da fome.

\label{section-10}

\chapter{O último
adeus}\label{capuxedtulo-xvi-o-uxfaltimo-adeus}

No dia em que Pauline Moser foi trazida ao hospício, ouvimos os gritos
mais terríveis que você pode imaginar. Uma menina irlandesa seminua veio
cambaleando pelo corredor como se estivesse bêbada. @--- Viva! Viva!
Matei o demônio! Lúcifer, Lúcifer, Lúcifer --- ela exclamava, e assim
por diante, sem parar, e então arrancava um tufo de cabelo. --- Ah, mas
eu enganei os demônios. Sempre disseram que Deus fez o inferno, mas não
fez, não. @ Pauline ajudou a menina a tornar o lugar ainda pior com suas
canções horríveis. Depois que a menina irlandesa estava lá havia mais ou
menos uma hora, o dr. Dent apareceu. Enquanto caminhava pelo corredor, a
Srta. Grupe cochichou para a menina demente: @--- Lá vem o demônio,
corre atrás dele. @ Surpresa que ela daria a uma louca essas instruções,
imaginei que a criatura histérica iria saltar sobre o doutor. Por sorte,
ela não o fez, apenas começou a repetir seu refrão de ``Lúcifer, oh,
Lúcifer''. Depois que o médico foi embora, a Srta. Grupe tentou
provocá-la mais uma vez, dizendo que o músico negro no quadro da parede
era o demônio. @--- Seu demônio, eu vou te pegar --- a pobre criatura
começou a berrar. @ Duas enfermeiras precisaram sentar sobre ela para
segurá-la. As atendentes pareciam obter prazer e diversão incitando as
pacientes violentas.

Sempre fiz questão de dizer aos médicos que era sã e pedir para ser
solta, mas quanto mais tentava garantir-lhes de minha sanidade, mais
eles duvidavam dela.

--- Por que vocês médicos estão aqui? --- perguntei a um deles, cujo
nome não lembro.

--- Para cuidar dos pacientes e testar sua sanidade --- ele respondeu.

--- Muito bem --- eu disse. --- Há dezesseis médicos nesta ilha, e com
exceção de dois, nunca vi nenhum deles dar atenção às pacientes. Como um
médico vai conseguir avaliar a sanidade de uma mulher se apenas lhe der
bom dia e se recusar a ouvir seus pedidos de liberdade? Mesmo as doentes
sabem que não adianta dizer nada, pois a resposta é sempre culpar sua
imaginação. @--- Tente todos os testes comigo --- eu insisti com outros.
--- E depois diga se sou sã ou se sou louca. Teste meu pulso, meu
coração, meus olhos, peça que estenda meu braço e que mexa meus dedos,
como o dr. Field fez em Bellevue, e depois diga se sou sã. @ Eles não me
davam atenção, pois achavam que estava delirando.

--- Vocês não têm o direito de manter pessoas sãs aqui --- eu disse para
outro. --- Eu sou sã, sempre fui e insisto em receber um exame completo
ou ser solta. Várias mulheres aqui também são sãs. Por que elas não
podem ser libertadas?

--- Elas são loucas --- foi a resposta. --- Estão sofrendo de delírios.

Depois de uma longa conversa com o dr. Ingram, ele disse que eu seria
transferida para uma ala mais quieta. Uma hora depois, a Srta. Grady me
chamou para o corredor e, depois de me chamar de todos os palavrões mais
vis e profanos que uma mulher consegue imaginar, disse que para a sorte
do meu ``couro'', eu estava sendo transferida, ou então pagaria caro por
contar tudo ao dr. Ingram. @--- Sua abusada dos infernos, você esquece
quem é, mas nunca esquece de contar tudo ao doutor. @ Depois de visitar
a Srta. Neville, que o dr. Ingram também tivera a bondade de transferir,
a Srta. Grady nos levou para o salão superior, de número 7.

\imagemmedia{}{./media/rId66.png}No
Salão 7 ficam a sra. Kroener e as Srtas. Fitzpatrick, Finney e Hart. Não
encontrei o mesmo tratamento cruel que no andar de baixo, mas ouvi elas
fazerem ameaças e observações cruéis e torcer os dedos e esbofetear as
pacientes que não se comportavam. A enfermeira noturna, creio que se
chama Conway, é muito mal-humorada. No Salão 7, se alguma das pacientes
possuía qualquer nível de modéstia, logo a perdeu. Todas eram forçadas a
se despir no corredor, em frente a suas próprias portas, e então dobrar
suas roupas e deixá-las ali até a manhã. Pedi para me despir dentro do
quarto, mas a Srta. Conway disse que se me pegasse com esse truque, me
ensinaria a nunca mais repeti-lo.

O primeiro médico que encontrei ali, o dr. Caldwell, me deu uma
pancadinha no queixo e, como estava cansada de me recusar a dizer onde
morava, só falei com ele em espanhol.

O Salão 7 parece simpático para um visitante ocasional. Ele é decorado
com quadros baratos e tem um piano, presidido pela Srta. Mattie Morgan,
que anteriormente trabalhava em uma loja de música na cidade. Ela
treinava várias pacientes na arte do canto, com algum sucesso. A artista
do salão é Under (pronuncia-se ``Wanda''), uma menina polonesa. É uma
pianista talentosa quando escolhe demonstrar suas habilidades. Ela é
capaz de ler as partituras mais difíceis em um instante e tem toque e
expressão perfeitos.

Aos domingos, as pacientes, mais calmas, cujos nomes foram marcados
pelas atendentes durante a semana, têm permissão de ir à igreja. A ilha
possui uma capela católica, e outros serviços religiosos também são
realizados.

Um ``comissário'' apareceu um dia e acompanhou o dr. Dent em sua ronda.
No porão, eles descobriram que metade das enfermeiras havia saído para
almoçar, deixando a outra metade para cuidar de nós, como era de
costume. Eles imediatamente mandaram que as enfermeiras voltassem a seus
deveres até após as pacientes terminarem de comer. Algumas das pacientes
quiseram falar sobre a falta de sal, mas foram impedidas.

O hospício da Ilha de Blackwell é uma ratoeira humana. É fácil entrar,
mas uma vez lá dentro, é impossível sair. Eu pretendia tentar ser
confinada a uma das alas violentas, a Cabana ou o Retiro, mas quando
obtive o testemunho de duas mulheres sãs, decidi não arriscar minha
saúde, e meu cabelo, então não fingi violência.

No final, eu fora isolada de todos os visitantes. Quando Peter A.
Hendricks, o advogado, veio e disse que meus amigos estavam dispostos a
se responsabilizar por mim se eu preferisse ficar com eles e não no
hospício, fiquei feliz em consentir. Pedi que ele encomendasse algo para
comer no instante em que chegasse na cidade e fiquei aguardando
ansiosamente por minha soltura.

Ela chegou mais cedo do que o esperado. Eu estava ``na fila'',
caminhando, e acabara de me interessar por uma pobre mulher que
desmaiara enquanto as enfermeiras tentavam forçá-la a caminhar. @---
Adeus, vou para casa --- gritei para Pauline Moser, enquanto estava
passava com uma mulher em cada lado. @ Infelizmente, disse adeus a todos
que conhecia a enquanto voltava para minha vida e para a liberdade,
deixando-as presas em um destino pior do que a morte. @--- \emph{Adiós}
--- murmurei para a mexicana, soprei um beijo para ela e abandonei
minhas companheiras do Salão 7.

Eu ficara tão ansiosa por sair daquele lugar horrível, mas quando chegou
a hora de minha soltura e soube que aproveitaria a luz divina em
liberdade mais uma vez, senti uma pontada de dor. Por dez dias, eu fora
uma delas. Pode soar estúpido, mas parecia extremamente egoísta
abandoná-las no sofrimento. Senti um desejo quixótico de ajudá-la com
solidariedade e com minha presença. Mas foi apenas por um momento. As
grades estavam derrubadas e a liberdade me era mais doce do que nunca.

Logo eu estava cruzando o rio e me aproximando de Nova Iorque, uma
mulher livre novamente depois de dez dias no manicômio da Ilha de
Blackwell.

\label{section-11}

\chapter{A investigação do
júri}\label{capuxedtulo-xvii-a-investigauxe7uxe3o-do-juxfari}

Pouco depois de dar adeus ao Hospício da Ilha de Blackwell, fui citada
para aparecer perante o júri de acusação. Respondi a citação com prazer,
pois ansiava por ajudar aquelas pobres filhas de Deus que haviam sido
deixadas prisioneiras quando parti. Se não podia lhes dar a dádiva
máxima da liberdade, tinha a esperança de ao menos conseguir influenciar
terceiros para que sua vida se tornasse mais suportável. Os jurados eram
todos cavalheiros e percebi que não precisaria tremer perante suas vinte
e três augustas presenças.

Jurei a verdade de minha história e contei tudo que me acontecera, desde
o início no Lar Temporário para Mulheres até minha soltura. O Assistente
de Promotoria Vernon M. Davis conduziu a inquirição. A seguir, os
jurados solicitaram que eu os acompanhassem em uma visita à ilha.
Consenti sem hesitar.

Ninguém deveria saber sobre a visita planejada à ilha, mas depois de
nossa chegada, não demorou para que um dos comissários de caridade e o
dr. MacDonald, da Ilha de Wards, se juntassem a nós. Um dos jurados me
disse que, conversando com um homem no hospício, ouviu que eles haviam
sido notificados de nossa vinda uma hora antes de chegarmos à ilha. Isso
deve ter acontecido enquanto os jurados examinavam o pavilhão dos
insanos em Bellevue.

A viagem até a ilha foi totalmente diferente da original. Desta vez,
fomos em um barco novo e limpo, enquanto aquele no qual eu viajara
estava, segundo nos informaram, encostado para consertos.

Algumas das enfermeiras foram inquiridas pelo júri e fizeram declarações
que contradiziam umas às outras e também a minha história. Elas
confessaram que a visita planejada dos jurados fora discutida entre elas
e com o doutor. O dr. Dent confessou que não tinha como ter certeza
absoluta se o banho era frio ou não ou o número de mulheres colocadas na
mesma água. Ele sabia que a comida não era tão boa quanto deveria, mas
disso que isso era explicado pela falta de fundos.

Ele tinha alguma maneira de determinar se as enfermeiras eram cruéis com
suas pacientes? Não, não tinha. Ele disse que nem todos os doutores eram
competentes, o que também se devia à falta de meios para obter médicos
de qualidade. Durante a conversa, ele me disse:

--- Fico feliz que você fez isso. Se soubesse qual era seu objetivo,
teria a ajudado. Não temos como descobrir a situação real das coisas,
exceto fazendo o que você fez. Desde que sua história foi publicada,
descobri uma enfermeira no Retiro que organiza vigias para alertar sobre
nossa chegada, tal qual você afirmou. Ela foi demitida.

A Srta. Anne Neville foi trazida e eu fui ao salão para encontrá-la,
sabendo que a visão de tantos cavalheiros estranhos a deixaria agitada,
apesar dela ser sã. Foi como eu temia. As atendentes haviam lhe
informado que ela seria examinada por um grande grupo de homens e ela
tremia de medo. Apesar de eu ter deixado a ala apenas duas semanas
antes, ela parecia ter sofrido uma doença grave no período, tão alterada
estava sua aparência. Perguntei se ela tomara algum remédio, ao que ela
respondeu que sim. Disse então que tudo que queria que fizesse era
contar ao júri tudo que havíamos feito desde que eu fora levada junto
com ela ao hospício, para que se convencessem da minha sanidade. Ela me
conhecia apenas como Nellie Brown e ignorava totalmente minha história
real.

Ela não foi compromissada como testemunha, mas sua história deve ter
convencido todos os ouvintes sobre a verdade de minhas afirmações.

--- Quando a Srta. Brown e eu fomos trazidas para cá, as enfermeiras
eram cruéis e a comida era ruim demais. Não tínhamos roupas o suficiente
e a Srta. Brown estava sempre pedindo mais. Achei ela muito bondosa,
pois quando um médico prometeu mais roupas para ela, a Srta. Brown disse
que as daria para mim. Por mais estranho que pareça, desde que a Srta.
Brown foi levada embora, tudo está diferente. As enfermeiras estão mais
gentis e todas temos o que vestir. Os médicos vêm nos ver mais vezes e a
comida melhorou bastante.

Ainda precisávamos de mais evidências?

Os jurados visitaram a cozinha. O espaço estava bastante limpo, com dois
barris de sal em posição de destaque junto à porta! O pão à mostra era
de uma brancura linda, completamente diferente daquele que nos fora dado
para comer.

Os salões e corredores estavam na mais perfeita ordem. As camas haviam
sido melhoradas. No salão 7, os baldes nos quais éramos forçadas a nos
lavar haviam sido substituídos por bacias novas em folha.

A instituição estava em exposição e era impossível encontrar qualquer
defeito.

Mas e as mulheres das quais falara, onde elas estavam? Nenhuma se
encontrava onde eu as deixara. Se minhas afirmações sobre essas
pacientes fossem falsas, por que elas teriam sido transferidas,
impedindo-me de encontrá-las? A Srta. Neville reclamou para o júri de
ter sido transferida várias vezes. Quando visitamos o salão mais tarde,
ela foi devolvida a seu lugar original.

Mary Hughes, que eu afirmara parecer sã, não foi encontrada. Alguns
parentes a haviam levado. Aonde, não sabiam dizer. A mulher de pele
clara que eu descrevera, enviada ao hospício por ser pobre, teria sido
transferida para outra ilha. Eles negaram qualquer conhecimento sobre a
mulher mexicana e disseram que jamais haviam tido uma paciente com essa
descrição. A sra. Cotter recebera alta, enquanto Bridget McGuinness e
Rebecca Farron haviam sido transferidas para outros aposentos. Margaret,
a menina alemã, não foi encontrada, enquanto Louise fora transferida do
salão 7. Josephine, a francesa, uma mulher saudável e encorpada, estaria
morrendo de paralisia, então não era possível vê-la. Se eu estava errada
em minha avaliação sobre a sanidade dessas pacientes, por que tudo isso
foi feito? Também vi Tillie Mayard, mas ela piorara tanto que me
arrepiei ao encontrá-la.

Eu não esperava que o júri fosse me amparar, tendo visto tudo diferente
da situação que encontrei em minha estadia. Mas foi o que fizeram, e seu
relatório para o tribunal aconselha a adoção de todas as minhas
propostas de mudança.

Meu trabalho me dá uma consolação: com base em minha história, o comitê
de apropriações alocou uma soma 1.000.000 de dólares maior do que jamais
fizera antes para o cuidado dos insanos.

\textbf{FIM}
